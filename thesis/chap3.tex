\chapter{Software Implementation}

\section{Overview}

My proposed MEng project involves building and analyzing a software
implementation that can make such inductive observations and
discoveries as described above.  Although there are many aspects
involved in replicating the human-like reasoning discussed, I will
initially focus on a core system that is able to model the inductive
exploratory behavior.  Further abilities can be added as extensions or
applications of the core system as described in section
\ref{sec:extension}.

This core system will provide an interpreter to accept input of
construction instructions, an analytic geometry system that can create
instances of such constructions, a pattern-finding process to discover
``interesting properties'', and an interface for reporting findings.

\subsection{Interpreting Construction Steps}

The first step in such explorations is interpreting an input of the
diagram to be explored.  To avoid the problems involved with solving
constraint systems and the possibility of impossible diagrams, the
core system will take as input an explicit list of construction steps
that results in an instance of the desired diagram.  These
instructions can still include arbitrary selections (let $P$ be some
point on the line, or let $A$ be some acute angle), but otherwise will
be restricted to basic construction operations using a compass and
straight edge.

To simplify the input of more complicated diagrams, some of these
steps can be abstracted into a library of known construction
procedures.  For example, although the underlying figures will be
limited to very simple objects of points, lines, and angles, the steps
of constructing a triangle (three points and three segments) or
bisecting a line or angle can be encapsulated into single steps.

\subsection{Creating Figures}

Given a language for expressing the constructions, the second phase of
the system will be to perform such constructions to yield an instance
of the diagram.  This process will mimic ``imagining'' manipulations
and will result in an analytic representation of the figure with
coordinates for each point.  Arbitrary choices in the construction
(``Let $Q$ be some point not on the line.'') will be chosen via an
random process, but with an attempt to keep the figures within a
reasonable scale to ease human inspection.

\subsection{Noticing Interesting Properties}
\label{sec:interest}

Having constructed a particular figure, the system will need to be
able to examine it to find interesting properties.  These properties
involve facts that appear to be ``beyond coincidence''.  As mentioned
in section \ref{sec:elem}, this generally involves relationships
between measured values, but can also include ``unexpected''
configurations of points, lines, and circles.  As the system discovers
interesting properties, it will reconstruct the diagram using
difference choices and observe if the observed properties hold true
across many instances of a diagram.

\subsection{Reporting Findings}

Finally, once the system has discovered some interesting properties
that appear repeatedly in instances of a given diagram, it will need a
means of reporting its results to the user.  Although this could
easily be a simple list of all simple relationships, some effort will
be taken to avoid repeating observations that obvious in the
construction.  For example, if a perpendicular bisector of segment
$AB$ is requested, the fact that it bisects that segment in every
instance is not informative.  To do so, the construction process will
also maintain a list of facts that can be reasoned from construction
assumptions so that these can be omitted in the final reporting.
