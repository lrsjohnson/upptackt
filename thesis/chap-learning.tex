\chapter{Learning Module}
\label{chap:learning}

\section{Overview}

As the final module, the learning module integrates information from
the other modules and provides the primary, top-level interface for
interacting with the system. It provides means for users to query its
knowledge and provide investigations for the system to carry
out. Through performing such investigations, the learning module
formulates conjectures based on its observations and maintains a
repository of information representing a student's understanding of
geometry concepts.

I will first discuss the interface for interacting with the
system. Then, after describing the structures for representing and
storing definitions and conjectures, I demonstrate how the system
module new terms and conjectures. Finally, I will explain the cyclic
interaction between the imperative and declarative modules used to
simplify definitions and discuss some limitations and future
extensions.

The Demonstration Chapter (Sections~\ref{sec:end-goal-1}
and~\ref{sec:end-goal-2}) included several use cases and examples of
working with the learning module. As a result, this discussion will
focus on structures and implementation rather than uses and
applications. Refer to the demonstration for examples.

%% Outline
%% - Interface (what-is, etc.) / Student
%% - Definitions and Conjectures, Lattice
%% - Learning terms and conjectures
%% - Simplifying definitions
%% - Discussion

\section{Learning Module Interface}

As seen in the demonstration, the learning module provides the primary
interface by which users interact with the system. As such, it
provides means by which users can both query the system to discover
and use what it has known, as well as to teach the system information
by suggesting investigations it should
undertake. Listing~\ref{l-interface} shows the implementation for some
of these methods.

\begin{code-example}
[label=l-interface]
{Learning System Interface Examples}
(define (what-is term)
  (pprint (lookup term)))

(define (example-object term)
  ((definition-generator (lookup term))))

(define (show-example term)
    (show-element (example-object term))

(define (is-a? term obj)
  (let ((def (lookup term)))
    (definition-holds? def obj)))

(define (examine object)
  (let ((satisfying-terms
         (filter (lambda (term) (is-a? term object))
           (known-terms))))
    (remove-supplants more-specific? satisfying-terms)))
\end{code-example}

Explaining these interface implementations provide a context for
introducing the representation of definitions and conjectures.
\enlargethispage*{\baselineskip}

\section{Querying}

Users can query the system's knowledge using \texttt{what-is}. When
queried, the system uses \texttt{lookup} to find a definition from its
dictionary. Printing this definition provides the classification (that
a rhombus is a parallelogram) and a set of properties that
differentiates that object from its classification. Further requests
can present all known properties of the named object or generate a
minimal set of properties needed to specify the object.

\subsection{Student Structure}

Internally, geometry knowledge is stored in a \texttt{student} object
that maintains a \texttt{dictionary} mapping terms to definitions and
a \texttt{term lattice} representing how these definitions relate to
one another. Listing~\ref{student-structure} demonstrates how the
interfaces above use a global \texttt{*current-student*} variable to
access information. Although the system currently only ever
instantiates one student, this architecture provides the flexibility
to teach or compare multiple students in the future.

\begin{code-listing}
[label=student-structure]
{Student Structure}
(define-record-type <student>
  (%make-student definition-dictionary term-lattice) ...)

(define (student-lookup-definition s name)
  (hash-table/get (student-dictionary s) name #f))

(define *current-student* (make-initialized-student))

(define (lookup-definition term)
  (student-lookup-definition *current-student* term))

(define (lookup term)
  (or (lookup-definition term) (error "Term Unknown:" term)))
\end{code-listing}

\subsection{Definition Structure}

\begin{code-listing}
[label=def-struct]
{Definition Structure}
(define-record-type <definition>
  (%make-definition name generator primitive-predicate
                     primitive?
                     all-conjectures
                     classifications specific-conjectures) ...)
\end{code-listing}

Listing~\ref{def-struct} shows the implementation of definition
structures. Definition combine the name and generator procedure
provided when originally learning the definition with a list of all
conjectures known about that class of object. \texttt{primitive?} is a
boolean indicator of whether the definition is a primitive, built-in
definition. In such cases, \texttt{base-predicate} is a imperative
system-level predicate that tests whether an object satisfies the
definition. In non-primitive definitions, the \texttt{base-predicate}
is that of the primitive that the definition is a specialization
of. Storing and checking against this primitive predicate prevents
inapplicable operations from being performed such as attempting to
obtain the angles of a segment object.

The last two fields, \texttt{classifications} and \texttt{specific
  conjectures} are derived fields that are updated based on the
definition's relation to other terms. A definition's
\texttt{classifications} are the next-least specific terms its class
of objects also satisfy and \texttt{specific-conjectures} are added
conjectures that differentiate the definition from being the union of
those classification definitions.


\section{Testing Definitions}

The learning module provides the \texttt{is-a?} procedure to test
whether a given object satisfies a known term. As shown in
Listing~\ref{def-holds}, testing whether a definition holds involves
ensuring that it is the right type of object by checking the
underlying primitive predicate and then ensuring the relevant
conjectures are satisfied.

In this nonrecursive version, the system checks that an object
satisfies all known conjectures, but a recursive version shown later
first checks that it satisfies the parent classifications before
checking definition-specific conjectures that differentiate it from
its classifications.

\begin{code-listing}
[label=def-holds]
{Definition Checking}
(define (definition-holds-nonrecursive? def obj)
  (let ((all-conjectures (definition-conjectures def)))
    (and ((definition-primitive-predicate def) obj)
         (every (lambda (conjecture)
                  (satisfies-conjecture? conjecture (list obj)))
                all-conjectures))))
\end{code-listing}

\subsection{Conjecture Structure}

Conjectures are similar to observations in that they associate a
perception relationship with information about what satisfies the
relationship. However, instead of associating a relationship with
actual elements that satisfy the relationship, conjectures abstract
this observation by storing only the symbolic dependencies and source
procedures of those arguments.

Similar to how Example~\ref{new-premise} in the imperative system used
the element source procedures to obtain constructed elements
corresponding to those observed in an original diagram, satisfying a
conjecture involves applying its source-procedures to a new premise
structure for the conjecture to obtain new relationship
arguments. These new arguments are then checked to see if they satisfy
the underlying relationship. This process is shown in
Listing\ref{conj-struct}. The interface procedure \texttt{is-a?}
creates a list of the object in question to use as the new premise.

\begin{code-listing}
[label=conj-struct]
{Conjecture Structure}
(define-record-type <conjecture>
  (make-conjecture dependencies source-procedures relationship) ...)

(define (satisfies-conjecture? conj premise-instance)
  (or (true? (observation-from-conjecture conj premise-instance))
      (begin (if *explain* (pprint `(failed-conjecture ,conj)))
             #f)))

(define (observation-from-conjecture conj premise-instance)
  (let ((new-args
         (map (lambda (construction-proc)
                (construction-proc premise-instance))
          (conjecture-construction-procedures conj)))
        (rel (conjecture-relationship conj)))
    (and (relationship-holds rel new-args)
         (make-observation rel new-args))))
\end{code-listing}

\section{Examining Objects}

\texttt{examine}, the last interface function shown in
Listing~\ref{l-interface}

\begin{code-listing}
[label=more-specific]
{Relations among terms}
(define (more-specific? more-specific-term less-specific-term)
  (let ((more-specific-obj (example-object more-specific-term)))
    (is-a? less-specific-term more-specific-obj)))
\end{code-listing}


\subsection{Maintaining the Term Lattice}

Discoveries are represented within a lattice of premises (discoveries
about quadrilaterals < discoveries about rhombuses < discoveries about
squares, but are separate from discoveries about circles or segments).

\begin{code-listing}
[label=updating-terms]
{Updating Terms from Lattice}
(define (add-definition-lattice-node! term)
  (add-lattice-node (definition-lattice) (make-lattice-node term term))
  (update-definitions-from-lattice (cons term (child-terms term))))

(define (update-definition-from-lattice term)
  (let* ((def (lookup term))
         (current-conjectures (definition-conjectures def))
         (ancestor-terms (ancestor-terms term))
         (ancestor-defs (map lookup ancestor-terms))
         (ancestor-conjectures
          (append-map definition-conjectures ancestor-defs))
         (new-conjectures
          (set-difference current-conjectures
                          ancestor-conjectures
                          conjecture-equivalent?)))
    (set-definition-classifications! def (parent-terms term))
    (set-definition-specific-conjectures! def new-conjectures)))
\end{code-listing}

\begin{code-listing}
[label=def-holds-2]
{Recursive Definition Holds}
(define (definition-holds? def obj)
  (let ((classifications (definition-classifications def))
        (specific-conjectures (definition-specific-conjectures def)))
    (and ((definition-predicate def) obj)
         (every (lambda (classification-term)
                  (is-a? classification-term obj))
                classifications)
         (every (lambda (conjecture)
                  (satisfies-conjecture? conjecture (list obj)))
                specific-conjectures))))
\end{code-listing}

\subsection{Core Knowledge}

\begin{code-listing}
[label=core-knowledge]
{Introducing Core Knowledge}
(define (provide-core-knowledge)
  (for-each add-definition! primitive-definitions))

(define primitive-definitions
  (list
   (make-primitive-definition 'object true-proc true-proc)
   (make-primitive-definition 'point point? random-point)
   (make-primitive-definition 'line line? random-line)
   ...
   (make-primitive-definition 'triangle triangle? random-triangle))
\end{code-listing}

\section{Learning new Terms and Conjectures}

To learn a new definition, the system must be given the name of the
term being learned as well as a procedure that will generate arbitrary
instances of that definition. To converge to the correct definition,
that random procedure should present a wide diversity of instances
(i.e. the random-parallelogram procedure should produce all sorts of
parallelograms, not just rectangles). However, reconciling mixed
information about what constitutes a term could be an interesting
extension.

\begin{code-listing}{Learning a new term}
(define (learn-term term object-generator)
  (if (term-known? term) (error "Term already known:" term))
  (let ((term-example (name-polygon (object-generator))))
    (let* ((primitive-predicate (get-primitive-predicate term-example))
           (fig (figure (as-premise term-example 0)))
           (observations (analyze-figure fig))
           (conjectures (map conjecture-from-observation observations)))
      (pprint conjectures)
      (let ((new-def
             (make-restrictions-definition term object-generator
                primitive-predicate conjectures)))
        (add-definition! new-def)
        (check-new-def new-def)
        'done))))


(define (conjecture-from-observation obs)
  (make-conjecture
   (map element-dependencies->list (observation-args obs))
   (map element-source (observation-args obs))
   (observation-relationship obs)))
\end{code-listing}

\subsection{Performing Investigations}

Investigations are similar to analyzing various figures above except
that they have the intent of the analysis results being placed in the
geometry knowledge repository. This separation also allows for
dependence information about where properties were derived from.

Given the lattice structure of definitions, an interesting question
when exploring new investigations is whether the given

\section{Simplifying Definitions}

To simplify definitions, we interface with the constraint solver
observations->figure to convert our observations back into a figure.

\begin{code-listing}{Simplifying Definitions}
(define (get-simple-definitions term)
  (let ((def (lookup term))
        (simple-def-result (make-simple-definitions-result)))
    (let* ((object ((definition-generator def)))
           (fig (figure (as-premise (name-polygon object) 0)))
           (all-observations (analyze-figure fig))
           (eligible-observations
            (filter observation->constraint all-observations)))
      (for-each
       (lambda (obs-subset)
         (if (simple-def-should-test? simple-def-result obs-subset)
             (let ((polygon
                    (polygon-from-object-observations object obs-subset)))
               ((cond ((false? polygon) mark-unknown-simple-def!)
                      ((is-a? term polygon)
                       (begin (pp "=> Sufficient")
                              mark-sufficient-simple-def!))
                      (else (begin (pp "=> Insufficient")
                                   mark-insufficient-simple-def!)))
                simple-def-result obs-subset)
               (simplify-definitions-result! simple-def-result))
             (pprint `(skipping ,obs-subset))))
       (shuffle (all-subsets eligible-observations)))
      (pprint simple-def-result)
      simple-def-result)))
\end{code-listing}

\begin{code-listing}{Converting Observations to a Figure}
(define (polygon-from-object-observations object obs-subset)
  (let* ((topology (topology-for-object object))
         (new-figure (observations->figure topology obs-subset)))
    (and new-figure (object-from-new-figure object new-figure))))

(define (establish-polygon-topology-for-polygon polygon)
  (let* ((points (polygon-points polygon))
         (vertex-names (map element-name points)))
    (apply m:establish-polygon-topology vertex-names)))

(define (observations->figure-one-trial topology observations)
  (initialize-scheduler)
  (let* ((constraints (observations->constraints observations))
         (m (m:mechanism topology constraints)))
    (m:build-mechanism m)
    (and (m:solve-mechanism m)
         (let ((fig (m:mechanism->figure m)))
           (show-figure fig)
           fig))
\end{code-listing}

\section{Discussion}

%% Old Stuff:
