\chapter{Learning Module}
\label{chap:learning}

\section{Overview}

The Learning and learning module is one the core elements integrating
information from the other components of the system. It maintains

\section{Interactions with Learning Module}

The learning module provides the primary interface by which users
interact with the system. As such, it provides means by which users
can both query the system to discover and use what it has known, as
well as to teach the system information by suggesting investigations
it should undertake.

\subsection{Querying}

A simple way of interacting with the learning module is to ask it for
what it knows about various geometry concepts or terms. For
definitions, the results provide the classification (that a rhombus is
a parallelogram), and a set of minimal properties that differentiates
that object from its classification. Further querying can present all
known properties of the named object as well as theorems involving
that term.

\begin{verbatim}
(what-is 'rhombus)
\end{verbatim}

\subsection{Learning Definitions}

To learn a new definition, the system must be given the name of the
term being learned as well as a procedure that will generate arbitrary
instances of that definition. To converge to the correct definition,
that random procedure should present a wide diversity of instances
(i.e. the random-parallelogram procedure should produce all sorts of
parallelograms, not just rectangles). However, reconciling mixed
information about what constitutes a term could be an interesting
extension.

\begin{verbatim}
(learn-term 'parallelogram random-parallelogram)
\end{verbatim}

\subsection{Applying Learned Properties}

To apply the learned conjectures, the learning system can use its
repository of geometry knowledge to tell you what a given figure is or
to point out old vs. new properties seen in a specific figure:

\begin{verbatim}
(analyze-figure figure)
(examine (random-rhombus))
\end{verbatim}

\subsection{Performing Investigations}

Investigations are similar to analyzing various figures above except
that they have the intent of the analysis results being placed in the
geometry knowledge repository. This separation also allows for
dependence information about where properties were derived from.

\section{Representing Discoveries}

Discoveries are represented within a lattice of premises (discoveries
about quadrilaterals < discoveries about rhombuses < discoveries about
squares, but are separate from discoveries about circles or segments).

\subsection{Placement of discoveries}

Given this lattice structure, an interesting question when exploring
new properties

\subsection{Ordering of discoveries}

An issue with this system is that often discoveries can be in slightly
different formats. As such, for each relationship, we establish a
consistent ordering of elements and use pattern matching to

For example, assertions about equality of segments |AB| = |CD| are
independent of the ordering of points within the elements.

\subsection{Pattern Matching against existing conjectures}

Based on dependencies, we replace the lowest-level random dependencies
with arbitrary pattern elements \texttt{(? s1 ,segment)} for
instance. Then, when new conjectures are being considered, we attempt
to pattern match based on existing elements to see if there is a
redundant observation.
