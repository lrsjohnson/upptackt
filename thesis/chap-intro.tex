\chapter{Introduction}
\label{chap:intro}

I developed and analyzed an interactive computer system that emulates
a student learning geometry concepts through inductive
investigation. Although geometry knowledge can be conveyed via a
series of factual definitions, theorems, and proofs, my system focuses
on a more investigative approach in which an external teacher guides
the student to ``discover'' new definitions and theorems via
explorations and self-directed inquiry.

My system emulates such a student by beginning with a fairly limited
knowledge set regarding basic definitions in geometry and providing a
means for a user interacting with the system to ``teach'' additional
geometric concepts and theorems by suggesting investigations the
system should explore to see if it ``notices anything interesting.''

To enable such learning, my project includes the combination of four
intertwined modules: an imperative geometry construction interpreter
to build constructions, a declarative geometry constraint solver to
solve and test specifications, an observation-based perception module
to notice interesting properties, and a learning module to analyze
information from the other modules and integrate it into new
definition and theorem discoveries.

To evaluate its recognition of such concepts, my system provides means
for a user to extract the observations and apply its findings to new
scenarios.  Through a series of simple investigations similar to an
introductory course in geometry, the system has been able to propose
and learn a few dozen standard geometry theorems.

\section{Document Structure}

Following this introduction,

\begin{description}

\item [Chapter~\ref{chap:motivation} Motivation] discusses motivation
  of the system and presents some examples of diagram manipulation,
  emphasizing the technique of visualizing diagrams ``in the mind's
  eye.''

\item [Chapter~\ref{chap:demo} Demonstration] provides several sample
  interactions with the system and introduces the general system
  components.

\item[Chapter~\ref{chap:sys-overview} System Overview] presents
  several concepts used in the system, introduces the four main
  modules, and discusses how they work together in the discovery of
  new definitions and theorems.

\item[Chapters~\ref{chap:imperative}~-~\ref{chap:learning}] describe
  the implementation and function of the four primary system modules:

\begin{description}
\item[Chapter~\ref{chap:imperative} Imperative Construction]
  describes the construction module that enables the system to
  represent, perform, and display constructions.

\item[Chapter~\ref{chap:observer} Perception] describes the perception
  module that focuses on observing interesting properties in
  diagrams. A key question involves determining ``what is
  interesting?''.

\item[Chapter~\ref{chap:declarative} Declarative Constraint Solver]
  describes the propagator-based declarative geometry constraint
  solver that builds instances of diagrams satisfying declarative
  constraints.

\item[Chapter~\ref{chap:learning} Learning Module] describes the
  learning module that integrates results from the other systems to
  create new discoveries. Main features include filtering out obvious
  or known results from investigations to focus on the most
  interesting discoveries, representing and storing newly-discovered
  definitions and theorems, and providing an interface to apply these
  findings to new situations.

\end{description}
\newpage

\item[Chapter~\ref{chap:related-work} Related Work] discusses some
  related work to automated geometry theorem discovery and proof, as
  well as a comparison with existing dynamic geometry systems.

\item[Chapter~\ref{chap:conclusion} Conclusion] evaluates the strengths and
  weaknesses of the system and discussed future work and possible extensions.

\item[Appendix~\ref{chap:code} Code Listings] provides full listings for code
  used in the system and explains an external dependency on a
  propagator system.

\item[Appendix~B Bibliography] lists works referenced in the document.

\end{description}
