% -*- Mode:TeX -*-

%% IMPORTANT: The official thesis specifications are available at:
%%            http://libraries.mit.edu/archives/thesis-specs/
%%
%%            Please verify your thesis' formatting and copyright
%%            assignment before submission.  If you notice any
%%            discrepancies between these templates and the
%%            MIT Libraries' specs, please let us know
%%            by e-mailing thesis@mit.edu

%% The documentclass options along with the pagestyle can be used to generate
%% a technical report, a draft copy, or a regular thesis.  You may need to
%% re-specify the pagestyle after you \include  cover.tex.  For more
%% information, see the first few lines of mitthesis.cls.

%\documentclass[12pt,vi,twoside]{mitthesis}
%%
%%  If you want your thesis copyright to you instead of MIT, use the
%%  ``vi'' option, as above.
%%
%\documentclass[12pt,twoside,leftblank]{mitthesis}
%%
%% If you want blank pages before new chapters to be labelled ``This
%% Page Intentionally Left Blank'', use the ``leftblank'' option, as
%% above.

\documentclass[12pt,twoside,bibliography=totoc]{mitthesis}
\usepackage{lgrind}
%% These have been added at the request of the MIT Libraries, because
%% some PDF conversions mess up the ligatures.  -LB, 1/22/2014
\usepackage{cmap}
\usepackage[T1]{fontenc}
\pagestyle{plain}

\usepackage{graphicx}


%% Code Listings:

\usepackage{multicol}
\usepackage{pdflscape}

\usepackage{listings}

\usepackage[T1]{fontenc}
\usepackage[scaled=0.85]{beramono}
\usepackage{listings}
\usepackage{textcomp}
\usepackage{color}
\usepackage{caption}

\usepackage[nottoc,notlof,notlot,numbib]{tocbibind}

\usepackage[utf8]{inputenc}

\usepackage{tcolorbox}
\tcbuselibrary{most}


\lstdefinestyle{mystyle}{numbers=left,numberstyle=\tiny,numbersep=5pt,
  morekeywords=[1]{define, define-syntax, define-macro, lambda, define-stream, stream-lambda},
  morekeywords=[2]{begin, call-with-current-continuation, call/cc,
    call-with-input-file, call-with-output-file, case, cond,
    do, else, for-each, if,
    let*, let, let-syntax, letrec, letrec-syntax,
    let-values, let*-values,
    and, or, not, delay, force,
    quasiquote, quote, unquote, unquote-splicing,
    map, fold, syntax, syntax-rules, eval, environment, query,
    let-geo*, define-record-type,
 },
  morekeywords=[3]{import, export},
  alsodigit=!\$\%&*+-./:<=>?@^_~,
  sensitive=true,
  morecomment=[l]{;},
  morecomment=[s]{\#|}{|\#},
  morestring=[b]",
  basicstyle=\footnotesize\ttfamily,
  keywordstyle=\bf\ttfamily,
  xleftmargin=0.1in,
  commentstyle=\color[rgb]{0.255,0.255,0.255},
  stringstyle={\color[rgb]{0.4,0.4,0.4}},
  upquote=true,
  breaklines=true,
  breakatwhitespace=true,
  literate=*{`}{{`}}{1},
  literate = {\%}{{{\%}}}2,
  showstringspaces=false,breaklines}

\lstset{style=mystyle}

%% Actual Code Listings:


\newtcblisting[auto counter,number within=chapter]{code-example}[2][]{
listing only, breakable,
code=\linespread{1}\vspace{1em},
listing options={style=mystyle, basicstyle=\footnotesize\ttfamily},
fonttitle=\bfseries,
title=Code Example \thetcbcounter: #2,#1}

\newtcblisting[use counter from=code-example]{repl-example}[2][]{
listing only, breakable,
code=\linespread{1}\vspace{1em},
listing options={style=mystyle, basicstyle=\footnotesize\ttfamily,
numbers=none},
fonttitle=\bfseries,
title=Interaction Example \thetcbcounter: #2,#1}

\newtcblisting[use counter from=code-example]{code-listing}[2][]{
listing only, breakable,
code=\linespread{1}\vspace{1em},
listing options={style=mystyle, basicstyle=\footnotesize\ttfamily},
fonttitle=\bfseries,
title=Code Listing \thetcbcounter: #2,#1}

\newtcblisting[use counter from=code-example]{img-example}[3][]{
listing and comment, righthand width=5cm,
code=\linespread{1}\vspace{1em},
listing options={style=mystyle,numbers=none,
basicstyle=\footnotesize\ttfamily},
tcbimage comment={#3},
fonttitle=\bfseries,
comment style={size=fbox,frame hidden,height=6cm},
%every listing line*={\textcolor{black}{\small\ttfamily\bfseries => }},
title=Interaction Example \thetcbcounter: #2,#1
}

\newtcblisting[use counter from=code-example]{pdf-example}[3][]{
listing and comment, righthand width=5cm, breakable,
code=\linespread{1}\vspace{1em},
listing options={style=mystyle,numbers=none,
basicstyle=\footnotesize\ttfamily},
pdf comment={#3},
fonttitle=\bfseries,
comment style={frame hidden,
opacityback=0,
height=6cm,
raster columns=2,graphics pages={1,2}},
%every listing line*={\textcolor{black}{\small\ttfamily\bfseries => }},
title=Interaction Example \thetcbcounter: #2,#1
}



%% End Code Listings

%% This bit allows you to either specify only the files which you wish to
%% process, or `all' to process all files which you \include.
%% Krishna Sethuraman (1990).

\typein [\files]{Enter file names to process, (chap1,chap2 ...), or `all' to
 process all files:}
\def\all{all}
\ifx\files\all \typeout{Including all files.} \else \typeout{Including only \files.} \includeonly{\files} \fi

\begin{document}

\include{cover}
% Some departments (e.g. 5) require an additional signature page.  See
% signature.tex for more information and uncomment the following line if
% applicable.
% \include{signature}
\pagestyle{plain}
\include{contents}
\chapter{Introduction}
\label{chap:intro}

I developed and analyzed an interactive computer system that emulates
a student learning geometry concepts through inductive
investigation. Although geometry knowledge can be conveyed via a
series of factual definitions, theorems, and proofs, my system focuses
on a more investigative approach in which an external teacher guides
the student to ``discover'' new definitions and theorems via
explorations and self-directed inquiry.

My system emulates such a student by beginning with a fairly limited
knowledge set regarding basic definitions in geometry and providing a
means for a user interacting with the system to ``teach'' additional
geometric concepts and theorems by suggesting investigations the
system should explore to see if it ``notices anything interesting.''

To enable such learning, my project includes the combination of four
intertwined modules: an imperative geometry construction interpreter
to build constructions, a declarative geometry constraint solver to
solve and test specifications, an observation-based perception module
to notice interesting properties, and a learning module to analyze
information from the other modules and integrate it into new
definition and theorem discoveries.

To evaluate its recognition of such concepts, my system provides means
for a user to extract the observations and apply its findings to new
scenarios.  Through a series of simple investigations similar to an
introductory course in geometry, the system has been able to propose
and learn a few dozen standard geometry theorems.

\section{Document Structure}

Following this introduction,

\begin{description}

\item [Chapter~\ref{chap:motivation} Motivation] discusses motivation
  of the system and presents some examples of diagram manipulation,
  emphasizing the technique of visualizing diagrams ``in the mind's
  eye.''

\item [Chapter~\ref{chap:demo} Demonstration] provides several sample
  interactions with the system and introduces the general system
  components.

\item[Chapter~\ref{chap:sys-overview} System Overview] presents
  several concepts used in the system, introduces the four main
  modules, and discusses how they work together in the discovery of
  new definitions and theorems.

\item[Chapters~\ref{chap:imperative}~-~\ref{chap:learning}] describe
  the implementation and function of the four primary system modules:

\begin{description}
\item[Chapter~\ref{chap:imperative} Imperative Construction]
  describes the construction module that enables the system to
  represent, perform, and display constructions.

\item[Chapter~\ref{chap:observer} Perception] describes the perception
  module that focuses on observing interesting properties in
  diagrams. A key question involves determining ``what is
  interesting?''.

\item[Chapter~\ref{chap:declarative} Declarative Constraint Solver]
  describes the propagator-based declarative geometry constraint
  solver that builds instances of diagrams satisfying declarative
  constraints.

\item[Chapter~\ref{chap:learning} Learning Module] describes the
  learning module that integrates results from the other systems to
  create new discoveries. Main features include filtering out obvious
  or known results from investigations to focus on the most
  interesting discoveries, representing and storing newly-discovered
  definitions and theorems, and providing an interface to apply these
  findings to new situations.

\end{description}
\newpage

\item[Chapter~\ref{chap:related-work} Related Work] discusses some
  related work to automated geometry theorem discovery and proof, as
  well as a comparison with existing dynamic geometry systems.

\item[Chapter~\ref{chap:conclusion} Conclusion] evaluates the strengths and
  weaknesses of the system and discussed future work and possible extensions.

\item[Appendix~\ref{chap:code} Code Listings] provides full listings for code
  used in the system and explains an external dependency on a
  propagator system.

\item[Appendix~B Bibliography] lists works referenced in the document.

\end{description}

\include{chap-motivation}
\chapter{Demonstration}
\label{chap:demo}

My system uses this idea of manipulating diagrams ``in the mind's
eye'' to explore and discover geometry theorems. Before discussing
some of the internal representations and modules, I will briefly
describe the goals of the system to provide direction and context to
understand the components.

\section{Imperative Figure Construction}

\begin{tcblisting}{colback=white,colframe=black!75!black,
listing only,
leftrule=3pt,
listing options={style=tcblatex,
basicstyle=\footnotesize\ttfamily}}
(define (triangle-with-pep-bisectors)
  (let-geo* ((a (make-point 0 0))
             (b (make-point 1.5 0))
             (c (make-point 1 1))
             (t (polygon-from-points a b c))
             (pb1 (perpendicular-bisector (make-segment a b)))
             (pb2 (perpendicular-bisector (make-segment b c)))
             (pb3 (perpendicular-bisector (make-segment c a))))
    (figure t pb1 pb2 pb3)))
\end{tcblisting}

\begin{tcblisting}{colback=white,colframe=black!75!black,
listing and comment, righthand width=5cm,
leftrule=3pt,
listing options={style=tcblatex,numbers=none,
basicstyle=\footnotesize\ttfamily},
tcbimage comment={images/demo-tri-pb-fig.png},
comment style={size=fbox,colframe=white,colback=white!50,height=8cm},
every listing line*={\textcolor{black}{\small\ttfamily\bfseries => }}}
(show-figure (triangle-with-perp-bisectors))
\end{tcblisting}



\begin{tcblisting}{colback=white,colframe=black!75!black,
listing only, righthand width=5cm,
leftrule=3pt,
fontlower=\ttfamily,
listing options={style=tcblatex,numbers=none,
basicstyle=\footnotesize\ttfamily},
comment style={size=fbox,colframe=white,colback=white!50,height=8cm}}
=> (show-figure (triangle-with-perp-bisectors))

((concurrent #[line 22] #[line 20] #[line 18])
 (perpendicular #[line 22] #[segment 21])
 (perpendicular #[line 20] #[segment 19])
 (perpendicular #[line 18] #[segment 17]))
\end{tcblisting}



\section{Declarative Constraint Solving}

\begin{lstlisting}[caption=Getting labels]
(define (arbitrary-triangle)
  (m:mechanism
   (m:establish-polygon-topology 'a 'b 'c)))
\end{lstlisting}

\begin{lstlisting}[caption=Constraint Solving for Isoceles Triangle]
(define (isoceles-triangle)
  (m:mechanism
   (m:establish-polygon-topology 'a 'b 'c)
   (m:c-length-equal (m:bar 'a 'b)
                     (m:bar 'b 'c))))
\end{lstlisting}


\begin{lstlisting}[caption=Constraint Solving for Isoceles Triangle]
(define (parallelogram-by-angles)
  (m:mechanism
   (m:establish-polygon-topology 'a 'b 'c 'd)
   (m:c-angle-equal (m:joint 'a)
                    (m:joint 'c))
   (m:c-angle-equal (m:joint 'b)
                    (m:joint 'd))))
\end{lstlisting}

\include{chap-sys-overview}
\chapter{Imperative Construction System}
\label{chap:imperative}

\section{Overview}

The first module is an imperative system for performing geometry
constructions. This is the typical input method for generating
coordinate-backed, analytic instances of figures.

The construction system is comprised of a large, versatile library of
useful utility and construction procedures for creating figures. To
appropriately focus the discussion of this module, I will concentrate
on the implementation of structures and procedures necessary for the
sample construction seen in Example \ref{sample-construction}.  Full
code and more usage examples are provided in Appendix \ref{chap:code}.

In doing so, I will first describe the basic structures and essential
utility procedures before presenting some higher-level construction
procedures, polygons, and figures. Then, I will Then, I will explore
the use of randomness in the system and examine how construction
language macros handle names, dependencies, and multiple assignment of
components.  Finally, I will briefly discuss the interface and
implementation for animating and displaying figures.

\begin{pdf-example}
[label=sample-construction]
{Sample Construction}
{images/angle-bisector-distance.pdf}
(define (angle-bisector-distance)
  (let-geo* (((a (r-1 v r-2)) (random-angle))
             (ab (angle-bisector a))
             (p (random-point-on-ray ab))
             ((s-1 (p b)) (perpendicular-to r-1 p))
             ((s-2 (p c)) (perpendicular-to r-2 p)))
     (figure a r-1 r-2 ab p s-1 s-2)))

=> (show-figure (angle-bisector-distance))
\end{pdf-example}

%% Outline
%% - Basic Structures and essential Utilities (code + line-from-points...)
%% - Randomness
%% - Higher-order Objects (Polygons, Figure)
%% - let-geo* Names and Dependencies
%% - Animation / Graphics.

\section{Basic Structures}

The basic structures in the imperative construction system are points,
segments, rays, lines, angles, and circles. These structures, as with
all structures in the system are implemented using Scheme record
structures as seen in Listings \ref{basic-structures} and
\ref{angle-circle-structs}. In the internal representations, lines and
segments are directioned. Predicates exist to allow other procedures
to work with or ignore these directions.

\begin{code-listing}
[label=basic-structures]
{Basic Structures}
(define-record-type <point>
  (make-point x y)
  point?
  (x point-x)
  (y point-y))

(define-record-type <segment>
  (%segment p1 p2)
  segment?
  (p1 segment-endpoint-1)
  (p2 segment-endpoint-2))

(define-record-type <line>
  (%make-line point dir)
  line?
  (point line-point) ;; Point on the line
  (dir line-direction))
\end{code-listing}

\begin{code-listing}
[label=angle-circle-structs]
{Angle and Circle Structures}
(define-record-type <angle>
  (make-angle dir1 vertex dir2)
  angle?
  (dir1 angle-arm-1)
  (vertex angle-vertex)
  (dir2 angle-arm-2))

(define-record-type <circle>
  (make-circle center radius)
  circle?
  (center circle-center)
  (radius circle-radius))
\end{code-listing}

\subsection{Creating Elements}

Elements can be created explicitly using the underlying
\texttt{make-*} constructors defined with the record types. However,
several higher-order constructors are provided to simplify
construction as shown in Listings~\ref{line-from-points}
and~\ref{angle-from}.  In \texttt{angle-from-lines}, we make use of
the fact that lines are directioned to uniquely specify an angle.  As
with the angle construction case, in several instances, we use generic
operations to handle mixed types of geometry elements.

\begin{code-listing}
[label=line-from-points]
{Higher-order Constructors}
(define (line-from-points p1 p2)
  (make-line p1 (direction-from-points p1 p2)))

\end{code-listing}

\begin{code-listing}
[label=angle-from]
{Generic Constructors for Creating Angles}
(define angle-from (make-generic-operation 2 'angle-from))

(define (angle-from-lines l1 l2)
  (let ((d1 (line->direction l1))
        (d2 (line->direction l2))
        (p (intersect-lines l1 l2)))
    (make-angle d1 p d2)))
(defhandler angle-from angle-from-lines line? line?)
\end{code-listing}

\subsection{Essential Math Utilities}

Several math utility structures support these constructors and other
geometry procedures. One particularly useful abstraction is a
\texttt{direction} that fixes a direction in the interval $[0, 2
  \pi]$. Listing~\ref{directions} provides a taste of some operations
using such abstractions.

\begin{code-listing}
[label=directions]
{Directions}
(define (subtract-directions d2 d1)
  (if (direction-equal? d1 d2)
      0
      (fix-angle-0-2pi (- (direction-theta d2)
                          (direction-theta d1)))))

(define (direction-perpendicular? d1 d2)
  (let ((difference (subtract-directions d1 d2)))
    (or (close-enuf? difference (/ pi 2))
        (close-enuf? difference (* 3 (/ pi 2))))))
\end{code-listing}

\section{Higher-order Procedures and Structures}

Higher-order construction procedures and structures are built upon
these basic elements and utilities. Listing~\ref{perpendiculars} shows
the implementation of the perpendicular constructions used in the
sample figure.

\begin{code-listing}
[label=perpendiculars]
{Perpendicular Constructions}
;; Constructs line through point perpendicular to linear-element
(define (perpendicular linear-element point)
  (let* ((direction (->direction linear-element))
         (rotated-direction (rotate-direction-90 direction)))
    (make-line point rotated-direction)))

;;; Constructs perpendicular segment from point to linear-element
(define (perpendicular-to linear-element point)
  (let ((pl (perpendicular linear-element point)))
    (let ((i (intersect-linear-elements pl (->line linear-element))))
      (make-segment point i))))
\end{code-listing}

Although traditional constructions generally avoid using rulers and
protractors, Listing~\ref{angle-bisector} shows the implementation of
the \texttt{angle-bisector} procedure from our sample figure that uses
measurements to simplify construction.

\begin{code-listing}
[label=angle-bisector]
{Angle Bisector Construction}
(define (angle-bisector a)
  (let* ((d2 (angle-arm-2 a))
         (vertex (angle-vertex a))
         (radians (angle-measure a))
         (half-angle (/ radians 2))
         (new-direction (add-to-direction d2 half-angle)))
    (make-ray vertex new-direction)))
\end{code-listing}

\subsection{Polygons and Figures}

Polygons record structures contain an ordered list of points in
counter-clockwise order, and provide procedures such as
\texttt{polygon-point-ref} or \texttt{polygon-segment} to obtain
particular points, segments, and angles specified by indices.

Figures are simple groupings of geometry elements and provide
procedures for extracting all points, segments, angles,
and lines contained in the figure, including ones extracted from within
polygons or subfigures.

\section{Random Choices}

To allow figures to represent general spaces of diagrams, random
choices are commonly used to instantiate diagrams. In our sample
figure, we use \texttt{random-angle} and \texttt{random-point-on-ray},
implementations of which are shown in
listing~\ref{random-angle-measure}. Underlying these procedures are
calls to Scheme's random function over a specified range ($[0, 2\pi]$
for \texttt{random-angle-measure}, for instance). Since infinite
ranges are not well supported and to ensure the figures stay
reasonable legible for a human viewer,
\texttt{extend-ray-to-max-segment} clips a ray at the current working
canvas so a point on the ray can be selected within the working
canvas.

\begin{code-listing}
[label=randomness]
{Random Constructors}
(define (random-angle)
  (let* ((v (random-point))
         (d1 (random-direction))
         (d2 (add-to-direction
              d1
              (rand-angle-measure))))
    (make-angle d1 v d2)))

(define (random-point-on-ray r)
  (random-point-on-segment
   (extend-ray-to-max-segment r)))

(define (random-point-on-segment seg)
  (let* ((p1 (segment-endpoint-1 seg))
         (p2 (segment-endpoint-2 seg))
         (t (safe-rand-range 0 1.0))
         (v (sub-points p2 p1)))
    (add-to-point p1 (scale-vec v t))))

(define (safe-rand-range min-v max-v)
  (let ((interval-size (max 0 (- max-v min-v))))
    (rand-range
     (+ min-v (* 0.1 interval-size))
     (+ min-v (* 0.9 interval-size)))))
\end{code-listing}

Other random elements are created by combining these random choices,
such as the random parallelogram in Listing
\ref{random-parallelogram}.

\begin{code-listing}
[label=random-parallelogram]
{Random Parallelogram}
(define (random-parallelogram)
  (let* ((r1 (random-ray))
         (p1 (ray-endpoint r1))
         (r2 (rotate-about (ray-endpoint r1)
                           (rand-angle-measure)
                           r1))
         (p2 (random-point-on-ray r1))
         (p4 (random-point-on-ray r2))
         (p3 (add-to-point
              p2
              (sub-points p4 p1))))
    (polygon-from-points p1 p2 p3 p4)))
\end{code-listing}

\subsection{Backtracking}

The module currently only provides limited support for avoiding
degenerate cases, or cases where randomly selected points happen to be
very nearly on top of existing points. Several random choices use
\texttt{safe-rand-range} (seen in Listing \ref{randomness}) to avoid
the edge cases of ranges, but further extensions could improve this
system to periodically check for unintended relationships and
backtrack to choose other values.

\section{Construction Language Support}

To simplify specification of figures, the module provides the
\texttt{leg-geo*} macro which allows for a multiple-assignment-like
extraction of components from elements and automatically tags
resulting elements with their variable names for future
reference. Listing~\ref{letgeo-expansion} shows the expansion of a
simple usage of \texttt{let-geo*} and
listing~\ref{component-assignment} shows some of the macros' implementation.

\begin{code-example}
[label=letgeo-expansion]
{Expansion of let-geo* macro}
(let-geo* (((a (r-1 v r-2)) (random-angle)))
  (figure a r-1 r-2 ...))

(let* ((a (random-angle))
       (r-1 (element-component a 0))
       (v   (element-component a 1))
       (r-2 (element-component a 2)))
  (set-element-name! a   'a)
  (set-element-name! r-1 'r-1)
  (set-element-name! v   'v)
  (set-element-name! r-2 'r-2)
  (figure a r-1 r-2 ...))
\end{code-example}

\begin{code-listing}
[label=component-assignment]
{Multiple and Component Assignment}
(define (expand-compound-assignment lhs rhs)
  (if (not (= 2 (length lhs)))
      (error "Malformed compound assignment LHS (needs 2 elements): " lhs))
  (let ((key-name (car lhs))
        (component-names (cadr lhs)))
    (if (not (list? component-names))
        (error "Component names must be a list:" component-names))
    (let ((main-assignment (list key-name rhs))
          (component-assignments (make-component-assignments
                                  key-name
                                  component-names)))
      (cons main-assignment
            component-assignments))))

(define (make-component-assignments key-name component-names)
  (map (lambda (name i)
         (list name `(element-component ,key-name ,i)))
       component-names
       (iota (length component-names))))
\end{code-listing}

Once expanded, a generic \texttt{element-component} operator shown in
Listing~\ref{element-component} defines what components are extracted
from what elements (endpoints for segments, vertices for polygons,
(ray, angle, ray) for angles).

\begin{code-listing}
[label=element-component]
{Generic Element Component Handlers}
(declare-element-component-handler polygon-point-ref polygon?)

(declare-element-component-handler
 (component-procedure-from-getters
  ray-from-arm-1
  angle-vertex
  ray-from-arm-2)
 angle?)
\end{code-listing}

\section{Graphics and Animation}

The system integrates with Scheme's graphics system for the X Window
System to display the figures for the users. The graphical viewer can
include labels and highlight specific elements, as well as display
animations representing the ``wiggling'' of the
diagram. Implementations of core procedures of these components are
shown in Listings~\ref{draw-figure} and~\ref{animation}.

\begin{code-listing}
[label=draw-figure]
{Drawing Figures}
(define (draw-figure figure canvas)
  (set-coordinates-for-figure figure canvas)
  (clear-canvas canvas)
  (for-each
   (lambda (element)
     (canvas-set-color canvas (element-color element))
     ((draw-element element) canvas))
   (all-figure-elements figure))
  (for-each
   (lambda (element)
     (canvas-set-color canvas (element-color element))
     ((draw-label element) canvas))
   (all-figure-elements figure))
  (graphics-flush (canvas-g canvas)))
\end{code-listing}

To animate a figure, constructions can call \texttt{animate} with a
procedure f that takes an argument in $[0, 1]$. When the animation is
run, the system will use fluid variables to iteratively wiggle each
successive random choice through its range of
$[0,1]$. \texttt{animate-range} provides an example where a user can
specify a range to wiggle over.

\begin{code-listing}
[label=animation]
{Animation}
(define (animate f)
  (let ((my-index *next-animation-index*))
    (set! *next-animation-index* (+ *next-animation-index* 1))
    (f (cond ((< *animating-index* my-index) 0)
             ((= *animating-index* my-index) *animation-value*)
             ((> *animating-index* my-index) 1)))))

(define (animate-range min max)
  (animate (lambda (v)
             (+ min
                (* v (- max min))))))
\end{code-listing}

%% Old Stuff:

\if false

\subsection{Points}

Points form the basis of most elements. Throughout the system, points
are labeled and used to identify other elements.

\subsection{Linear Elements}

The linear elements of Segments, Lines, and Rays are built upon
points. Initially the internal representation of lines were that of
two points, but to simplify manipulations,

To better specify angles (see below), all linear elements, including
segments and lines are directioned. Thus, a line pointing. Predicates
exist that compare lines for equality ignoring

\subsection{Angles}

Initially angles were represented as three points, now vertex + two
directions. CCW orientation. Methods exist to determine them from
various pairs of linear elements, uses directionality of linear
elements to determine which ``quadrant'' of the angle is desired.

Given a figure, methods exist to extract angles from the diagrams in
analysis rather than specifying each angle of interest while creating
the diagram.

\section{Higher-level structures}

In addition to the basic geometry structures, the system uses several
grouping structures to combine and abstract the basic figure elements
into higher-level figures elements.

For closure of combinators, all these higher level objects are also
``Diagram objects''.

\subsection{Polygons}

Polygons are represented as groups of points.

\subsection{Figures}

Figures are currently groups of elements. In the creation of figures
we extract additional information and build a graph out of adjacent
components for use in the analysis stages.

\section{Construction Operations}

\subsection{Traditional constructions}

\begin{code-listing}{Perpendiculars}
(define (perpendicular linear-element point)
  (let* ((direction (->direction linear-element))
         (rotated-direction (rotate-direction-90 direction)))
    (make-line point rotated-direction)))

;;; endpoint-1 is point, endpoint-2 is on linear-element
(define (perpendicular-to linear-element point)
  (let ((pl (perpendicular linear-element point)))
    (let ((i (intersect-linear-elements pl (->line linear-element))))
      (make-segment point i))))

(define (angle-bisector a)
  (let* ((d1 (angle-arm-1 a))
         (d2 (angle-arm-2 a))
         (vertex (angle-vertex a))
         (radians (angle-measure a))
         (half-angle (/ radians 2))
         (new-direction (add-to-direction d2 half-angle)))
    (make-ray vertex new-direction)))
\end{code-listing}

Midpoint, perpendicular line, bisectors

\subsection{Intersections}

Generic intersections, mathematically based at line/line or
line/circle at the core. Other intersections also add the check that
the resulting point(s) are on the elements.

\subsection{Measurement-based operations}

A ``Ruler + Protractor'' is generally not permitted in traditional
construction problems. However, sometimes its nice to be able to use
measurements to more quickly compute a result (e.g.\ angle bisector by
halving angle) vs.\ going through the whole ray/circle based
construction process.

\subsection{Transformations}

Currently, rotate about a point or translate by a vector. Also
interfaces for by *random* point or vector.

\section{Randomness}

\subsection{Random Choices}

At the basis of all random

\subsection{Remembering choices}

\subsection{Backtracking}

Currently, the system does not backtrack based on random
choices. However, there are plans to perform checks on
randomly-generated elements that are too close to one another and to
retry the random choice to avoid degenerate choices.

\subsection{Avoiding almost-degenerate points}

As discussed above, randomly making choices in

\subsection{Animating choices}

I animate over a small range within the specified random
range. Top-level infrastructure determinies frames, sleeping, etc.
Constructions can request to animate functions of one arg [0, 1]. As
the figure and animation is run, each call to randomize gets a call to
random whenever their value is non-false.

\section{Dependencies}

\subsection{Implementation}

Eq-properties, etc.

\subsection{Naming}

Sometimes derived if unknown, figure out how name metadata relates to
the dependencies.

\subsection{Forcing higher-level random dependenceis}

``Inverts'' the dependency tree that would otherwise usually go
down to points. Or set-dependency! as random-square. When given an
element by the teacher, generally we don't know how the construction
was performed.

\subsection{Dependency-less diagrams}

In some cases, the dependency structure of a figure can be wiped.

\section{Construction Language}

Constructions and instruction-based investigations are specified by
scheme procedures that return the desired figures.

\subsection{Macros}

I created a let-geo* special form that is similar to Scheme's (let
\ldots) form, but sets the element names as specified so they can be more
easily referred to later.

\subsection{Multiple Assignment}

In let-geo*, I also permit some constructions to optionally map to
multiple assignments of names, such as the case in which you create a
triangle and simulatneously want to store and name the triangle's
vertex points.

\section{Graphics}

The system integrates with Scheme's graphics system for the X Window
System to display the figures for the users. The graphical viewer can
include labels and highlight specific elements, as well as display
animations representing the ``wiggling'' of the diagram.

\begin{code-listing}{Drawing}
(define (draw-figure figure canvas)
  (set-coordinates-for-figure figure canvas)
  (clear-canvas canvas)
  (for-each
   (lambda (element)
     (canvas-set-color canvas (element-color element))
     ((draw-element element) canvas))
   (all-figure-elements figure))
  (for-each
   (lambda (element)
     (canvas-set-color canvas (element-color element))
     ((draw-label element) canvas))
   (all-figure-elements figure))
  (graphics-flush (canvas-g canvas)))
\end{code-listing}

\section{Discussion}

\fi

\chapter{Perception Module}
\label{chap:observer}

\section{Overview}

Given a module that executes construction steps to build analytic
figures, we need a way of ``seeing'' these figures in our mind's eye.
Thus, the perception module is primarily concerned with the task of
examining the figure and observing interesting properties in figure.

\begin{code-listing}{Analyzer Routine}
(define (analyze figure)
  (let* ((points (figure-points figure))
         (angles (figure-angles figure))
         (linear-elements (figure-linear-elements figure))
         (segments (figure-segments figure)))
    (append
     (extract-relationships points
                            (list concurrent-points-relationship
                                  concentric-relationship
                                  concentric-with-center-relationship))
     (extract-relationships segments
                             (list equal-length-relationship))
     (extract-relationships angles
                             (list equal-angle-relationship
                                   supplementary-angles-relationship
                                   complementary-angles-relationship))
     (extract-relationships linear-elements
                             (list parallel-relationship
                                   concurrent-relationship
                                   perpendicular-relationship)))))
\end{code-listing}

\begin{code-listing}
(define (extract-relationship elements relationship)
  (map (lambda (tuple)
         (make-observation relationship tuple))
       (report-n-wise
        (relationship-arity relationship)
        (relationship-predicate relationship)
        elements)))

(define (report-n-wise n predicate elements)
  (let ((tuples (all-n-tuples n elements)))
    (filter (nary-predicate n predicate) tuples)))
\end{code-listing}

\subsection{Extracting segments and angles}

The observation module also builds and traverses a
graph-representation of the object of connectedness and adjacencies to
extract more segments and angles, or include intersections of elements
in its investigation.

\subsubsection{Auxillary Segments}

In some circumstances, the system can insert and consider segments
between all pairs of points. Although this can sometimes produce
interesting results, it can often lead to too many elements being
considered. This option is off by default but can be enabled in a
self-exploration mode.

\subsection{What is Interesting?}

Concurrent points, collinear points, equal angles,
supplementary/complementary angles, parallel, perpendicular elements,
concentric points, (future:) ratios between measurements, etc.

\begin{code-listing}{Relationships}
(define-record-type <relationship>
  (%make-relationship type arity predicate)
  relationship?
  (type relationship-type)
  (arity relationship-arity)
  (predicate relationship-predicate))

(define equal-length-relationship
  (%make-relationship 'equal-length 2 segment-equal-length?))

(define concurrent-relationship
  (%make-relationship 'concurrent 3 concurrent?))
\end{code-listing}

\subsection{Removing Obvious Properties}

This module makes use of available dependency information to eliminate
some obvious properties. At this phase, the eliminations arise only
from basic geometry knowledge ``hard-coded'' into the system, and not
upon any specific prior-learned formula.

\subsubsection{Trivial relations}

Points being on lines, segments, circles directly dependent on that point.

\subsubsection{Branch Relations}

Other examples include ``branch'' relations. [REF: Chen, Song,
  etc.]. ABCD on a line with AB = CD also means that AC = BD, for instance.

\section{Representations}

A ``Relationship'' object represents a type of relationship, a
``Observation'' object refers to a specific observation seen in a figure.

\chapter{Declarative Geometry Constraint Solver}
\label{chap:declarative}

\section{Overview}

The third module is a declarative geometry constraint solver. Given a
user-specified topology of a diagram and various constraints on
segments and angles, this module attempts to solve the specification
by instantiating a figure that satisfies the constraints.

The solver is implemented using propagators, uses new types of partial
information about point regions and direction intervals, and focuses
on emulating the mental process of wiggling constrained figures in the
mind's eye. The physical nature of this process is captured by forming
analogies between geometry diagrams and mechanical linkages of bars
and joints.

After providing a brief overview of the mechanical analogies and quick
background on the propagator system, I examine an example of the
system solving a set of constraints for an under-constrained
rectangle. Then, I describe the module implementation, starting with
the new partial information representations and linkage constraints
before explaining how mechanisms are assembled and solved. Finally,
some limitations and extensions are discussed.

%% Outline
%% - Mechanical Analogies
%% - Propagator System
%% - Partial Information Structures
%% - Linkages
%% - Building a Mechanism
%% - Solving a Mechanism
%% - Discussion

\section{Mechanical Analogies}

Mechanical analogies are often applied to mathematical problems to
yield alternate, sometimes more-intuitive solutions. Several texts
such as [??], [??], and [??] study this and provide examples.

In this system, mechanical analogies are used to represent the physics
simulation going on as one mentally manipulates a diagram ``in the
mind's eye''. Often, given a diagram with constraints, one can imagine
assembling a physical example of the figure out of bars and joints in
one's head.  Some bars can be sliding to make their lengths adjustable
whereas others are constrained to be of equal length as one another.
As a person moves and wiggles these pieces to assemble satisfying
mechanisms, they can examine whether the resulting mechanisms retain
properties across instances and generalize such invariants into
theorems.

This module simulates this process by assembling mechanisms of bars
and joints, and using a propagator system to simulate incrementally
selecting where bars and joints are positioned while maintaining local
physical constraints.

\section{Propagator System}

The declarative geometry solver is built upon an existing propagator
system created by Alexey Radul under the advisement of Gerald Jay
Sussman \cite{gjs-propagator}. The propagator system allows a user to
create cells and connect them with propagator constraints. As content
is added to cells, their neighbors are notified and updated with
computations performed on the new information. Often, cells maintain a
representation of partial information about their content and merge
new information from several sources.

This module uses Radul's propagation system to handle the underlying
propagation of data, but implements constraints, partial information
types, specification protocols, and input formats particular to geometric
figures.

\section{Example of Solving Geometric Constraints}
\label{sec:example-solving}

I begin by fully explaining an example. The geometry problem of
inadequately constrained rectangles was introduced in the first
example of Chapter~\ref{chap:motivation} on page
\pageref{example-1}. The second proposed set of constraints in that
problem was expressed as a mechanism in Example~\ref{is-rect-2} in the
demonstration (page \pageref{is-rect-2}), and is repeated here in
Example~\ref{is-rect-specs}. Example~\ref{is-rect-solved} shows the
module's print messages as it solves the mechanism.

The illustrations in Explanation~\ref{illustration} and accompanying text
on the following pages explain how propagation is used to solve this
mechanism.

\begin{code-example}
[label=is-rect-specs]
{Rectangle Constraints Example}
(define (is-this-a-rectangle-2)
  (m:mechanism
   (m:establish-polygon-topology 'a 'b 'c 'd)
   (m:c-length-equal (m:bar 'a 'd) (m:bar 'b 'c))
   (m:c-right-angle (m:joint 'd))
   (m:c-angle-equal (m:joint 'a) (m:joint 'c))))
\end{code-example}

\begin{img-example}
[label=is-rect-solved]
{Solved Constraints}{images/rect-demo-2.png}
=> (m:run-mechanism (is-this-a-rectangle-2))

(specifying-bar m:bar:d:a .6742252545577186)
(initializing-direction m:bar:d:a-dir (direction 4.382829365403101))
(initializing-point m:bar:d:a-p1 (0 0))
(specifying-joint m:joint:c:b:a 2.65583669872538)
\end{img-example}

Solving a mechanism involves repeatedly selecting positions, lengths,
angles, and directions that are not fully specified and selecting
values within the domain of that element's current partial
information. As values are specified, the wiring of the propagator
model propagates further partial information to other values.

\refstepcounter{tcb@cnt@code-example}\label{illustration}
\begin{figure}[h]
\captionsetup{labelformat=empty}
\caption{{\bf Propagation Explanation
    \arabic{chapter}.\arabic{tcb@cnt@code-example}:} This series of
  illustrations depicts the propagation steps that occur to enable the
  system to solve the underconstrained rectangle from
  Example~\ref{is-rect-specs}.
}
\centering
\includegraphics[width=.95\textwidth]{diagrams/is-rect-explained-boards-1.eps}
\caption{{\bf Step 1:} The first value the module specifies is the
  length of bar \texttt{ad}. In doing so, it also initializes the
  bar's endpoint and direction to anchor it on the canvas. Because
  joint \texttt{d} is constrained to be a right angle, the system knows
  the direction but not length of bar \texttt{dc}. It propagates the
  partial information that point \texttt{c} is on the ray $r1$ extending
  out from \texttt{d} to the cell within point
  \texttt{c}. Furthermore, since bars \texttt{ad} and \texttt{bc} are
  constrained to have equal length, at this point, bar \texttt{bc} also knows
  its length but not direction. Next, the system specifies joint angle \texttt{b}:}
\includegraphics[width=.95\textwidth]{diagrams/is-rect-explained-boards-2.eps}
\caption{{\bf Step 2:} Once the angle measure of \texttt{b} is
  specified, constraints using the sum of angles in the specified
  polygon and a ``slice'' constraint on the pair of constrained angles
  will set the angle measures of joints \texttt{a} and \texttt{c} to
  be half of the remaining total:
  $\text{\texttt{a,c}}\protect\leftarrow\frac{2\protect\pi -
    \text{\texttt{b}} - \text{\texttt{d}}}{2}$. With these angles
  specified, point \texttt{b} is informed that it is on the ray $r2$
  and bar \texttt{bc} now knows both its length and direction.}
\vspace{-10em}
\end{figure}

\newpage
\begin{figure}[h]
\captionsetup{labelformat=empty}
\caption{{\bf Propagation Explanation
    \arabic{chapter}.\arabic{tcb@cnt@code-example} continued:} This series of
  illustrations depicts the propagation steps that occur to enable the
  system to solve the underconstrained rectangle solved in
  Example~\ref{is-rect-specs}.}
\centering
\includegraphics[width=.95\textwidth]{diagrams/is-rect-explained-boards-3.eps}
\caption{{\bf Step 3:} Since now both the length and direction of bar
  \texttt{bc} are known and point \texttt{c} is known to be on ray
  $r1$, the propagation constraints can translate this ray by the
  length and direction of \texttt{bc} and provide the information that
  point \texttt{b} must therefore also be on ray $r3$. This emulates
  the physical process of sliding bar \texttt{bc} along ray $r1$.}
\includegraphics[width=.95\textwidth]{diagrams/is-rect-explained-boards-4.eps}
\caption{{\bf Step 4:} The information about point \texttt{b} being on
  rays $r2$ and $r3$ is merged via ray intersection to fully determine
  the location of \texttt{b}. Then, once point \texttt{b} is
  specified, since the length and direction of bar \texttt{bc} is
  known, propagation sets the value and location of point \texttt{c},
  yielding a fully-specified solution.}
\vspace{-1.5em}
\end{figure}
Similar steps allow propagation to solve specifications for many
figures including isoceles triangles, parallelograms, and
quadrilaterals from their diagonals. Several of these are shown in
Section \ref{demo:sec:declarative}. In cases when bars have their
length and one endpoint specified first, the propagators specify that
the other endpoint is on an arc of a circle. The next sections
describe the implementation of these partial information structures
before explaining linkages and how mechanisms are built and solved.

\newpage
\section{Partial Information Structures}

Radul's propagation system typically used numeric intervals for
partial information. The declarative constraint solver uses standard
some numeric intervals, but also

\subsection{Regions}

Propagating partial information across bars and joints yields a new
region system: Regions include point sets of one or more possible
points, an entire ray, or an entire arc. These rays and arcs are
from an anchored bar with only one of direction or length specified,
for instance.

\begin{code-listing}
[label=regions]
{Region Structures}
(define-record-type <m:point-set>
  (%m:make-point-set points)
  m:point-set? ...)

(define-record-type <m:arc>
  (m:make-arc center-point radius dir-interval)
  m:arc? ...)

(define-record-type <m:ray>
  (%m:make-ray endpoint direction)
  m:ray? ...)

(define-record-type <m:region-contradiction>
  (m:make-region-contradiction error-regions)
  m:region-contradiction? ...)
\end{code-listing}

\subsection{Direction Intervals}

Ranges of intervals. Full circle + invalid intervals. Adding and
subtracting intervals of direction and thetas gets complicated at times.

Challenges with intersection, multiple segments. Eventually just
return nothing is okay.

\section{Linkages and Basic Constraints}

The solver uses bar and joint linkages.

Bars have endpoints, directions and length. Joints have a vertex point
and two directions. Currently, most joints are directioned and have
max value of 180 degrees.

\begin{code-listing}
[label=point-region]
{Points and Regions}
(define (m:make-point)
  (let-cells (x y region)
    (p:m:x-y->region x y region)
    (p:m:region->x region x)
    (p:m:region->y region y)
    (%m:make-point x y region)))

(define (m:x-y->region x y)
  (m:make-singular-point-set (make-point x y)))
(propagatify m:x-y->region)

(define (m:region->x region)
  (if (m:singular-point-set? region)
      (point-x (m:singular-point-set-point region))
      nothing))
(propagatify m:region->x)
(propagatify m:region->y)
\end{code-listing}

\begin{code-listing}
[label=bar-struct]
{Vectors}
(define (m:make-vec)
  (let-cells (dx dy length direction)
    (p:make-direction (e:atan2 dy dx) direction)
    (p:sqrt (e:+ (e:square dx)
                 (e:square dy))
            length)
    (p:* length (e:direction-cos direction) dx)
    (p:* length (e:direction-sin direction) dy)
    (%m:make-vec dx dy length direction)))
\end{code-listing}

\begin{code-listing}
[label=bar-struct]
{Basic Bar Structure}
(define (m:make-bar bar-id)
  (let ((p1 (m:make-point))
        (p2 (m:make-point))
        (v (m:make-vec)))
    (c:+ (m:point-x p1) (m:vec-dx v)
         (m:point-x p2))
    (c:+ (m:point-y p1) (m:vec-dy v)
         (m:point-y p2))
    (let ((bar (%m:make-bar p1 p2 v)))
      (m:p1->p2-bar-propagator p1 p2 bar)
      (m:p2->p1-bar-propagator p2 p1 bar)
      bar)))
\end{code-listing}

\begin{code-listing}
[label=bar-propagator]
{Bar Propagator}
(define (m:p1->p2-bar-propagator p1 p2 bar)
  (let ((p1x (m:point-x p1))
        (p1y (m:point-y p1))
        (p1r (m:point-region p1))
        (p2r (m:point-region p2))
        (length (m:bar-length bar))
        (dir (m:bar-direction bar)))
    (p:m:x-y-direction->region p1x p1y dir p2r)
    (p:m:x-y-length-di->region p1x p1y length dir p2r)
    (p:m:region-length-direction->region p1r length dir p2r)))

(define (m:x-y-length-di->region px py length dir-interval)
  (if (direction-interval? dir-interval)
      (let ((vertex (make-point px py)))
        (m:make-arc vertex length dir-interval))
      nothing))
(propagatify m:x-y-length-di->region)
\end{code-listing}

\subsection{Joint Constraints}
\begin{code-listing}
{Joint Constraints}
(define (m:make-joint)
  (let ((vertex (m:make-point)))
    (let-cells (dir-1 dir-2 theta)
      (p:m:add-to-direction dir-1 theta dir-2)
      (p:m:add-to-direction dir-2 (e:negate theta) dir-1)
      (p:m:subtract-directions dir-2 dir-1 theta)
      (m:instantiate theta (make-interval 0 *max-joint-swing*) 'theta)
      (%m:make-joint vertex dir-1 dir-2 theta))))
\end{code-listing}

\section{User-specified Constraints}

\begin{code-listing}{User Constraints}
(define-record-type <m:constraint>
  (m:make-constraint type args constraint-procedure)
  m:constraint?
  (type m:constraint-type)
  (args m:constraint-args)
  (constraint-procedure m:constraint-procedure))

(define (m:c-length-equal bar-id-1 bar-id-2)
  (m:make-constraint
   'm:c-length-equal
   (list bar-id-1 bar-id-2)
   (lambda (m)
     (let ((bar-1 (m:lookup m bar-id-1))
           (bar-2 (m:lookup m bar-id-2)))
       (c:id (m:bar-length bar-1)
             (m:bar-length bar-2))))))
\end{code-listing}

Angle sum of polygon, or scan through polygon and ensure that the
angles don't not match. Example is equilateral triangle, for
instance... Could also observe always ``60 degrees'' as an interesting
fact and put that in as a constraint. They're alebgraically quite
similar, but my propagators currently don't perform symbolic algebra.

\subsection{Slices}

\begin{code-listing}
[label=sum-slice]
{Sum Slice}
(define (m:equal-values-in-sum equal-cells all-cells total-sum)
  (let ((other-values (set-difference all-cells equal-cells eq?)))
    (c:id (car equal-cells)
          (ce:/ (ce:- total-sum (ce:multi+ other-values))
                (length equal-cells)))))

(define (m:sum-slice elements cell-transformer equality-predicate total-sum)
  (let* ((equivalence-classes
          (partition-into-equivalence-classes elements equality-predicate))
         (nonsingular-classes (filter nonsingular? equivalence-classes))
         (all-cells (map cell-transformer elements)))
    (cons (c:id total-sum (ce:multi+ all-cells))
          (map (lambda (equiv-class)
                 (m:equal-values-in-sum
                  (map cell-transformer equiv-class) all-cells total-sum))
               equivalence-classes))))
\end{code-listing}

\begin{code-listing}
[label=poly-sum-slice]
{Polygon Sum Slice}
(define (m:polygon-sum-slice all-joint-ids)
  (m:make-slice
   (m:make-constraint 'm:joint-sum all-joint-ids
    (lambda (m)
      (let ((all-joints (m:multi-lookup m all-joint-ids))
            (total-sum (n-gon-angle-sum (length all-joint-ids))))
        (m:joints-constrained-in-sum all-joints total-sum))))))

(define (m:joints-constrained-in-sum all-joints total-sum)
  (m:sum-slice all-joints m:joint-theta
   m:joints-constrained-equal-to-one-another? total-sum))
\end{code-listing}

\section{Building Mechanisms}

The Mechanism in our declarative system is analogous to Figure,
grouping elements. Also computes various caching and lookup tables to
more easily access elements.

\begin{code-listing}
[label=mechanism-struct]
{Mechanism Structure}
(define-record-type <m:mechanism>
    (%m:make-mechanism bars joints constraints slices
                       bar-table joint-table joint-by-vertex-table)
    m:mechanism? ...)

(define (m:mechanism . args)
  (let ((elements (flatten args)))
    (let ((bars (m:dedupe-bars (filter m:bar? elements)))
          (joints (filter m:joint? elements))
          (constraints (filter m:constraint? elements))
          (slices (filter m:slice? elements)))
      (m:make-mechanism bars joints constraints slices))))
\end{code-listing}

\begin{code-listing}
[label=est-topo]
{Establishing Topology}
(define (m:establish-polygon-topology . point-names)
  (if (< (length point-names) 3)
      (error "Min polygon size: 3"))
  (let ((extended-point-names
         (append point-names (list (car point-names) (cadr point-names)))))
    (let ((bars (map (lambda (p1-name p2-name)
                       (m:make-named-bar p1-name p2-name))
                     point-names
                     (cdr extended-point-names)))
          (joints (map (lambda (p1-name vertex-name p2-name)
                         (m:make-named-joint p1-name vertex-name p2-name))
                       (cddr extended-point-names)
                       (cdr extended-point-names)
                       point-names)))
      (append bars joints
              (list (m:polygon-sum-slice (map m:joint-name joints)))))))
\end{code-listing}

\begin{code-listing}
{Building Mechanisms}
(define (m:build-mechanism m)
  (m:identify-vertices m)
  (m:assemble-linkages (m:mechanism-bars m)
                       (m:mechanism-joints m))
  (m:apply-mechanism-constraints m)
  (m:apply-slices m))
\end{code-listing}

\begin{code-listing}
[label=identifying-points]
{Identifying points}
(define (m:identify-into-arm-1 joint bar)
  (m:set-joint-arm-1 joint bar)
  (m:identify-points (m:joint-vertex joint)
                     (m:bar-p2 bar))
  (c:id (ce:reverse-direction (m:joint-dir-1 joint))
        (m:bar-direction bar)))

(define (m:identify-points p1 p2)
  (for-each (lambda (getter)
              (c:id (getter p1)
                    (getter p2)))
            (list m:point-x m:point-y m:point-region)))
\end{code-listing}

\section{Solving Mechanisms}

\begin{code-listing}
[label=solve-mechanism]
{Solving Mechanisms}
(define (m:solve-mechanism m)
  (m:initialize-solve)
  (let lp ()
    (run)
    (cond ((m:mechanism-contradictory? m)
           (m:draw-mechanism m c)
           #f)
          ((not (m:mechanism-fully-specified? m))
           (if (m:specify-something m)
               (lp)
               (error "Couldn't find anything to specify.")))
          (else 'mechanism-built))))
\end{code-listing}

\begin{code-listing}
[label=specify-something]
{Specifying and Instantiating Values}
(define (m:specify-something m)
  (or
   (m:specify-bar-if m m:constrained?)
   (m:specify-joint-if m m:constrained?)
   (m:specify-joint-if m m:joint-anchored-and-arm-lengths-specified?)
   (m:initialize-bar-if m m:bar-length-specified?)
   ...)

(define (m:instantiate cell value premise)
  (add-content cell
    (make-tms (contingent value (list premise)))))
\end{code-listing}


Given a wired diagram, process is repeatedly specifying values for elements


\subsection{Interfacing with existing diagrams}

Converts between figures and symbolic relationships.

\begin{code-listing}
[label=to-figure]
{Converting to Figure}
(define (m:mechanism->figure m)
  (let ((points (map m:joint->figure-point (m:mechanism-joints m)))
        (segments (map m:bar->figure-segment (m:mechanism-bars m)))
        (angles (map m:joint->figure-angle (m:mechanism-joints m))))
    (apply figure (filter identity (append points segments angles)))))
\end{code-listing}

\section{Discussion and Extensions}

Future efforts involve an improved backtrack-search mechanism if
constraints fail, and a system of initializing the diagram with
content from an existing figure, kicking out and wiggling arbitrary
premises, and seeing how the resulting diagram properties respond.

\subsection{Backtracking}

If it can't build a figure with a given set of specifications, it will
first try some neighboring values, then backtrack and try a new value
for the previous element. After a number of failed attempts, it will
abort and claim that at this time, it is unable to build a diagram
satisfying the constraints.

\chapter{Learning Module}
\label{chap:learning}

\section{Overview}

As the final module, the learning module integrates information from
the other modules and provides the primary, top-level interface for
interacting with the system. It provides means for users to query its
knowledge and provide investigations for the system to carry
out. Through performing such investigations, the learning module
formulates conjectures based on its observations and maintains a
repository of information representing a student's understanding of
geometry concepts.

I will first discuss the interface for interacting with the
system. Then, after describing the structures for representing and
storing definitions and conjectures, I demonstrate how the system
module new terms and conjectures. Finally, I will explain the cyclic
interaction between the imperative and declarative modules used to
simplify definitions and discuss some limitations and future
extensions.

The Demonstration Chapter (Sections~\ref{sec:end-goal-1}
and~\ref{sec:end-goal-2}) included several use cases and examples of
working with the learning module. As a result, this discussion will
focus on structures and implementation rather than uses and
applications. Refer to the demonstration for examples.

%% Outline
%% - Interface (what-is, etc.) / Student
%% - Definitions and Conjectures, Lattice
%% - Learning terms and conjectures
%% - Simplifying definitions
%% - Discussion

\section{Learning Module Interface}

As seen in the demonstration, the learning module provides the primary
interface by which users interact with the system. As such, it
provides means by which users can both query the system to discover
and use what it has known, as well as to teach the system information
by suggesting investigations it should
undertake. Listing~\ref{l-interface} shows the implementation for some
of these methods.

\begin{code-example}
[label=l-interface]
{Learning System Interface Examples}
(define (what-is term)
  (pprint (lookup term)))

(define (example-object term)
  ((definition-generator (lookup term))))

(define (show-example term)
    (show-element (example-object term))

(define (is-a? term obj)
  (let ((def (lookup term)))
    (definition-holds? def obj)))

(define (examine object)
  (let ((satisfying-terms
         (filter (lambda (term) (is-a? term object))
           (known-terms))))
    (remove-supplants more-specific? satisfying-terms)))
\end{code-example}

Explaining these interface implementations provide a context for
introducing the representation of definitions and conjectures.
\enlargethispage*{\baselineskip}

\section{Querying}

Users can query the system's knowledge using \texttt{what-is}. When
queried, the system uses \texttt{lookup} to find a definition from its
dictionary. Printing this definition provides the classification (that
a rhombus is a parallelogram) and a set of properties that
differentiates that object from its classification. Further requests
can present all known properties of the named object or generate a
minimal set of properties needed to specify the object.

\subsection{Student Structure}

Internally, geometry knowledge is stored in a \texttt{student} object
that maintains a \texttt{dictionary} mapping terms to definitions and
a \texttt{term lattice} representing how these definitions relate to
one another. Listing~\ref{student-structure} demonstrates how the
interfaces above use a global \texttt{*current-student*} variable to
access information. Although the system currently only ever
instantiates one student, this architecture provides the flexibility
to teach or compare multiple students in the future.

\begin{code-listing}
[label=student-structure]
{Student Structure}
(define-record-type <student>
  (%make-student definition-dictionary term-lattice) ...)

(define (student-lookup-definition s name)
  (hash-table/get (student-dictionary s) name #f))

(define *current-student* (make-initialized-student))

(define (lookup-definition term)
  (student-lookup-definition *current-student* term))

(define (lookup term)
  (or (lookup-definition term) (error "Term Unknown:" term)))
\end{code-listing}

\subsection{Definition Structure}

\begin{code-listing}
[label=def-struct]
{Definition Structure}
(define-record-type <definition>
  (%make-definition name generator primitive-predicate
                     primitive?
                     all-conjectures
                     classifications specific-conjectures) ...)
\end{code-listing}

Listing~\ref{def-struct} shows the implementation of definition
structures. Definition combine the name and generator procedure
provided when originally learning the definition with a list of all
conjectures known about that class of object. \texttt{primitive?} is a
boolean indicator of whether the definition is a primitive, built-in
definition. In such cases, \texttt{base-predicate} is a imperative
system-level predicate that tests whether an object satisfies the
definition. In non-primitive definitions, the \texttt{base-predicate}
is that of the primitive that the definition is a specialization
of. Storing and checking against this primitive predicate prevents
inapplicable operations from being performed such as attempting to
obtain the angles of a segment object.

The last two fields, \texttt{classifications} and \texttt{specific
  conjectures} are derived fields that are updated based on the
definition's relation to other terms. A definition's
\texttt{classifications} are the next-least specific terms its class
of objects also satisfy and \texttt{specific-conjectures} are added
conjectures that differentiate the definition from being the union of
those classification definitions.


\section{Testing Definitions}

The learning module provides the \texttt{is-a?} procedure to test
whether a given object satisfies a known term. As shown in
Listing~\ref{def-holds}, testing whether a definition holds involves
ensuring that it is the right type of object by checking the
underlying primitive predicate and then ensuring the relevant
conjectures are satisfied.

In this nonrecursive version, the system checks that an object
satisfies \emph{all} known conjectures. A recursive version shown
later first checks that it satisfies the parent classifications before
checking definition-specific conjectures that differentiate it from
its classifications.

\begin{code-listing}
[label=def-holds]
{Definition Checking}
(define (definition-holds-nonrecursive? def obj)
  (let ((all-conjectures (definition-conjectures def)))
    (and ((definition-primitive-predicate def) obj)
         (every (lambda (conjecture)
                  (satisfies-conjecture? conjecture (list obj)))
                all-conjectures))))
\end{code-listing}

\subsection{Conjecture Structure}

Conjectures are similar to observations in that they associate a
perception relationship with information about what satisfies the
relationship. However, instead of associating a relationship with
actual elements that satisfy the relationship, conjectures abstract
this observation by storing only the symbolic dependencies and source
procedures of those arguments.

Similar to how Example~\ref{new-premise} in the imperative system used
the element source procedures to obtain constructed elements
corresponding to those observed in an original diagram, satisfying a
conjecture involves applying its source-procedures to a new premise
structure for the conjecture to obtain new relationship
arguments. These new arguments are then checked to see if they satisfy
the underlying relationship. This process is shown in
Listing~\ref{conj-struct}. The interface procedure \texttt{is-a?}
creates a list of the object in question to use as the new premise.

\begin{code-listing}
[label=conj-struct]
{Conjecture Structure}
(define-record-type <conjecture>
  (make-conjecture dependencies source-procedures relationship) ...)

(define (satisfies-conjecture? conj premise-instance)
  (or (true? (observation-from-conjecture conj premise-instance))
      (begin (if *explain* (pprint `(failed-conjecture ,conj)))
             #f)))

(define (observation-from-conjecture conj premise-instance)
  (let ((new-args
         (map (lambda (construction-proc)
                (construction-proc premise-instance))
          (conjecture-construction-procedures conj)))
        (rel (conjecture-relationship conj)))
    (and (relationship-holds rel new-args)
         (make-observation rel new-args))))
\end{code-listing}

\section{Examining Objects}

Given these tests, \texttt{examine}, the last interface function shown
in Listing~\ref{l-interface} allows a user to provide a geometry
object and ask the system to examine it and report what it is. Its
implementation (in Listing~\ref{l-interface}) first determines all
terms that apply to an object and then removes terms that are
supplanted by others in the list. It uses the procedure
\texttt{more-specific?} to determine which terms supplant others. As
shown in Listing~\ref{more-specific}, this procedure generates checks
if an example object of the proposed less specific term satisfies the
definition of the proposed more specific term.

\begin{code-listing}
[label=more-specific]
{Relations among terms}
(define (more-specific? more-specific-term less-specific-term)
  (let ((more-specific-obj (example-object more-specific-term)))
    (is-a? less-specific-term more-specific-obj)))
\end{code-listing}


\subsection{Maintaining the Term Lattice}

In addition to helping remove redundant information in results, this
partial order on terms is used to build and maintain a lattice of
terms in the student structure. This lattice can be rendered to a
figure using dot/Graphviz as shown in Example~\ref{full-lattice-pic}.

\begin{img-example}
[label=full-lattice-pic,
breakable=false,
comment style={size=fbox,frame hidden,height=8cm}]
{Full Definition Lattice}{images/full-lattice.png}
=> (show-definition-lattice)
\end{img-example}

The lattice is implemented as a general lattice data structure I
created that can be used with any partial order comparator. It
correctly positions nodes and updates the relevant parent and child
pointers as nodes are added and removed.

Information from the lattice is used to update the derived fields of
terms. As seen in Listing~\ref{updating-terms}, after a new definition
term is added to the lattice, it and its child terms (determined from
lattice) are updated. The immediate parent nodes in the lattice become
the definitions classifications, and definition-specific conjectures
are the set difference of all their conjectures and the conjectures
known about their ancestors in the lattice.

\begin{code-listing}
[label=updating-terms]
{Updating Terms from Lattice}
(define (add-definition-lattice-node! term)
  (add-lattice-node (definition-lattice) (make-lattice-node term term))
  (update-definitions-from-lattice (cons term (child-terms term))))

(define (update-definition-from-lattice term)
  (let* ((def (lookup term))
         (current-conjectures (definition-conjectures def))
         (ancestor-terms (ancestor-terms term))
         (ancestor-defs (map lookup ancestor-terms))
         (ancestor-conjectures
          (append-map definition-conjectures ancestor-defs))
         (new-conjectures
          (set-difference current-conjectures
                          ancestor-conjectures
                          conjecture-equivalent?)))
    (set-definition-classifications! def (parent-terms term))
    (set-definition-specific-conjectures! def new-conjectures)))
\end{code-listing}

This lattice structure allows terms definitions to build off of one
another and allows definitions to report only . These updated
classification and definition-specific properties are also used in the
full version of checking when a definition holds as shown in
Listing~\ref{def-holds-2}. This version checks that a definition
satisfies all parent classifications first before checking the
definition-specific conjectures that differentiate it from those
classifications.

\begin{code-listing}
[label=def-holds-2]
{Recursive Definition Holds}
(define (definition-holds? def obj)
  (let ((classifications (definition-classifications def))
        (specific-conjectures (definition-specific-conjectures def)))
    (and ((definition-predicate def) obj)
         (every (lambda (classification-term)
                  (is-a? classification-term obj))
                classifications)
         (every (lambda (conjecture)
                  (satisfies-conjecture? conjecture (list obj)))
                specific-conjectures))))
\end{code-listing}

\subsection{Core Knowledge}

To initialize the system, the student structure is provided with
several primitive definitions at startup as shown in
Listing~\ref{core-knowledge}.

\begin{code-listing}
[label=core-knowledge]
{Introducing Core Knowledge}
(define (provide-core-knowledge)
  (for-each add-definition! primitive-definitions))

(define primitive-definitions
  (list
   (make-primitive-definition 'object true-proc true-proc)
   (make-primitive-definition 'point point? random-point)
   (make-primitive-definition 'line line? random-line)
   ...
   (make-primitive-definition 'triangle triangle? random-triangle))
\end{code-listing}

\section{Learning new Terms and Conjectures}

To learn a new definition, the system must be given the name of the
term being learned as well as a procedure that will generate arbitrary
instances of that definition. To converge to the correct definition,
that random procedure should present a wide diversity of instances
(i.e. the random-parallelogram procedure should produce all sorts of
parallelograms, not just rectangles). However, reconciling mixed
information about what constitutes a term could be an interesting
extension.

\begin{code-listing}
[label=learn-term]
{Learning a new term}
(define (learn-term term object-generator)
  (if (term-known? term) (error "Term already known:" term))
  (let ((term-example (name-polygon (object-generator))))
    (let* ((primitive-predicate (get-primitive-predicate term-example))
           (fig (figure (as-premise term-example 0)))
           (observations (analyze-figure fig))
           (conjectures (map conjecture-from-observation observations)))
      (pprint conjectures)
      (let ((new-def
             (make-definition term object-generator
                primitive-predicate conjectures)))
        (add-definition! new-def)
        (check-new-def new-def)
        'done))))

(define (conjecture-from-observation obs)
  (make-conjecture
   (map element-dependencies->list (observation-args obs))
   (map element-source (observation-args obs))
   (observation-relationship obs)))
\end{code-listing}

Listing~\ref{learn-term} shows the implementation of the
\texttt{learn-term} procedure. It uses the provided generator
procedure to produce an example object for the term, creates a figure
with that object as its premise and obtains observations. These
observations are converted to conjectures via
\texttt{conjecture-from-observation} and the resulting definition is
added to the student dictionary and term lattice.

\subsection{Performing Investigations}

As demonstrated in Example~\ref{diag-investigation} (page
\pageref{diag-investigation}), the learning module also supports
investigations to learn conjectures based on elements constructed from
the base element.  Performing investigations are similar to learning
terms except that, rather than providing a procedure that just
generates an example of the term in consideration, an investigation
uses a procedure which takes an instance of the premise (polygon in
these cases) and constructs an entire figure to analyze. In addition
to reporting the interesting observations of such investigations,
conjectures for new observations derived by that investigation are
added to the definition for the term under investigation.

\section{Simplifying Definitions}

As properties accumulate from analysis and investigation, the need to
satisfy all known properties for a shape overconstraints the resulting
definitions. Thus, the final role of the learning module is to
simplify term definitions by checking declarative constraints.

As seen in Listing~\ref{simple-def}, \texttt{get-simple-definitions}
takes a known term, looks up the known properties for that term, and
tests all reasonable subsets of those properties as constraints using
the constraint solver. For each subset of properties, if the
constraint solver was able to create a diagram satisfying exactly
those properties, the resulting diagram is checked using with the
\texttt{is-a?} procedure to see if all the other known properties of
the original term still hold.

If so, the subset of properties is reported as a sufficient definition
of the term, and if the resulting diagram fails some properties, the
subset is reported as an insufficient set of constraints. These
resulting sufficient definitions can be treated as equivalent, simpler
definitions and used as the premises in new theorems about the
objects.

\begin{code-listing}
[label=simple-def]
{Simplifying Definitions}
(define (get-simple-definitions term)
  (let ((def (lookup term))
        (simple-def-result (make-simple-definitions-result)))
    (let* ((object ((definition-generator def)))
           (fig (figure (as-premise (name-polygon object) 0)))
           (all-observations (analyze-figure fig))
           (eligible-observations
            (filter observation->constraint all-observations)))
      (for-each
       (lambda (obs-subset)
         (if (simple-def-should-test? simple-def-result obs-subset)
             (let ((polygon
                    (polygon-from-object-observations object obs-subset)))
               ((cond ((false? polygon) mark-unknown-simple-def!)
                      ((is-a? term polygon) mark-sufficient-simple-def!)
                      (else mark-insufficient-simple-def!))
                simple-def-result obs-subset)
               (simplify-definitions-result! simple-def-result))
             (pprint `(skipping ,obs-subset))))
       (shuffle (all-subsets eligible-observations)))
      simple-def-result)))
\end{code-listing}

The \texttt{simple-definitions-result} structure maintains information
about what subsets are known to sufficient or insufficient as the
analysis proceeds and provides the predicate
\texttt{simple-def-should-test?} to skip over subsets where the result
is already known.

The main workhorse in this definition simplification process is the
procedure \texttt{polygon-from-object-observations}. It interfaces
with the constraint solver via observations->figure to convert our
observations back into a figure. Its implementation is shown below in
Listing~\ref{convert-obs}. The object provided is used to determine
the topology and names of bars and linkages in the mechanism and the
objects in the observation structures in the provided observation
subset is used to add the necessary mechanism constraints. If the
declarative system can solve the mechanism, it once again uses the
element names to extract the resulting object and return the a figure.

\begin{code-listing}
[label=convert-obs]
{Converting Observations to a Figure}
(define (polygon-from-object-observations object obs-subset)
  (let* ((topology (topology-for-object object))
         (new-figure (observations->figure topology obs-subset)))
    (and new-figure (object-from-new-figure object new-figure))))

(define (establish-polygon-topology-for-polygon polygon)
  (let* ((points (polygon-points polygon))
         (vertex-names (map element-name points)))
    (apply m:establish-polygon-topology vertex-names)))

(define (observations->figure-one-trial topology observations)
  (initialize-scheduler)
  (let* ((constraints (observations->constraints observations))
         (m (m:mechanism topology constraints)))
    (m:build-mechanism m)
    (and (m:solve-mechanism m)
         (let ((fig (m:mechanism->figure m)))
           (show-figure fig)
           fig))
\end{code-listing}

\newpage
\section{Discussion}

The learning module has been able to successfully integrate with the
other system modules to discover and learn dozens of simple elementary
geometry terms and theorems through its investigations. These include
simple properties such as ``the base angles in an isoceles triangle
are congruent,'' derived properties such as ``the diagonals of a
rhombus are orthogonal and bisect one another'' or ``the polygon found
by connecting consecutive side midpoints of an orthodiagonal
quadrilateral is always a rectangle,'' and simplified definitions such
as ``a quadrilateral with two pairs of congruent opposite angles is a
parallelogram.''

The current system has focused on discoveries related to
polygons. Further extensions of the module could explore ideas related
to other object types (segments, lines, circles) or derive conjectures
that depend on several arbitrary choices. Finally, an interesting
extension of the learning module would be to investigation properties
about constructions ``this is how you create a perpendicular
bisector'' so that the system could infer what interesting properties
such constructions yield and omit those observations when that
construction is used.

\chapter{Related Work}
\label{chap:related-work}

The topics of automating geometric proofs and working with diagrams
are areas of active research.  Several examples of related work can be
found in the proceedings of annual conferences such as \emph{Automated
  Deduction in Geometry} \cite{autoDeduction} and \emph {Diagrammatic
  Representation and Inference} \cite{diagramInference}.  In addition,
two papers from the past year combine these concepts with a layer of
computer vision interpretation of diagrams.  Chen, Song, and Wang
present a system that infers what theorems are being illustrated from
images of diagrams \cite{fromImages}, and a paper by Seo and
Hajishirzi describes using textual descriptions of problems to improve
recognition of their accompanying figures \cite{diagramUnderstanding}.

Further related work includes descriptions of the educational impacts
of dynamic geometry approaches and some software to explore geometric
diagrams and proofs.  However, such software typically uses alternate
approaches to automate such processes, and few focus on inductive
reasoning.

\section{Dynamic Geometry}
From an education perspective, there are several texts that emphasize
an investigative, conjecture-based approach to teaching such as
\emph{Discovering Geometry} by Michael Serra \cite{serraDiscovering},
the text I used to learn geometry and that served as an inspiration to
this thesis project.  Some researchers praise these investigative
methods \cite{geoTransformations} while others question whether they
appropriately encourages deductive reasoning skills
\cite{geoTeaching}.

\section{Software}
Some of these teaching methods include accompanying software such as
Cabri Geometry \cite{cabri} and the Geometer's Sketchpad
\cite{geoSketchpad} designed to enable students to explore
constructions interactively.  These programs occasionally provide
scripting features, but have no proof-related automation.

A few more academic analogs of these programs introduce some proof
features.  For instance, GeoProof \cite{geoProof} integrates diagram
construction with verified proofs using a number of symbolic methods
carried out by the Coq Proof Assistant, and Geometry Explorer
\cite{geoExplorer} uses a full-angle method of chasing angle relations
to check assertions requested by the user.  However, almost none of
the software described simulates the exploratory, inductive
investigation process used by students first discovering new
conjectures.

The closest example is Geometer

\section{Mechanical Analogies to Geometry Constraint Solving}

Books about Mechanical Analogies

\section{Automated Proof and Discovery}
Although there are several papers that describe automated discovery or
proof in geometry, most of these use alternate, more algebraic methods
to prove theorems.  These approaches include an area method
\cite{autoTools}, Wu's Method involving systems of polynomial
equations \cite{wuMethod}, and a system based on Gr\"obner Bases
\cite{grobner}.  Some papers discuss reasoning systems including the
construction and application of a deductive database of geometric
theorems \cite{deductiveDatabase}.  However, all of these methods
focused either on deductive reasoning or complex algebraic
reformulations.

\section{Other Resources}

GJS / Radul Propagators, ghelper code, etc.

%\chapter{Results}
\label{chap:results}

\section{Overview}

\chapter{Conclusion}
\label{chap:conclusion}

\section{Overview}

The system presented in this thesis provides a versatile framework for
building, exploring, and analyzing geometry diagrams. As shown in the
demonstrations, the modules can both be used independently to
construct and analyze interesting properties in geometric figures, and
combined with one another to discover new geometry concepts. By
constructing and examining figures, generalizing observations, solving
constraints, and aggregating results, the system has been able to
discover, learn, and simplify dozens of elementary geometry properties
and theorems.

In doing so, the process modeled and emulated the human-like process
of imagining and manipulating instance of problems ``in the mind's
eye'' to better understand new concepts. By focusing on noticing
interesting invariants in externally specified investigations, it
simulates the effectiveness of an investigative-based approach to
learning and discovering geometry concepts.

Although the architecture of the four interrelated imperative
construction building, perceiving, declarative constraint solving, and
learning modules serves as a proof of concept of and foundation for
exploring such a learning approach, it has room for further
improvement and extension. Several chapters conclude with a discussion
section including ideas for future extensions and improvements.

In addition, while the techniques developed in this system generally
reflect my own approach to visualizing and thinking about geometry and
background in learning geometry via an investigative approach, there
is room to integrate the \emph{discovery} ideas in this system with
some of the techniques from the rich history of automated geometry
theorem \emph{proving}.

\section{Limitations}

Despite its successes, there are certainly limitations to the system's
current abilities. Reasoning about geometry concepts is a very broad
domain, and it becomes difficult to develop general techniques that
can apply in a wide variety of circumstances.
Chapters~\ref{chap:observer} and~\ref{chap:declarative} discuss how
this challenge arises when trying to filter more categories of obvious
observations and when deciding the ideal method for specifying values
in the constraint solver.  There are also some sizable limitations to
the system's purely-investigative approach that restrict what it is
able to discover:

\subsection{Probabilistic Approach}

One challenge is that its approach is inherently probabilistic. As
with any numerical-based system, an important issue with using a
coordinate-based, inductive technique for discovering concepts is
dealing with numerical inaccuracies. Although techniques were used to
lessen some of the effects of floating point errors, such techniques
also emphasize the probabilistic nature of the system.  Without using
deductive reasoning, the system cannot ever fully confirm its findings
are correct and may occasionally report false properties due to
uncertainty. However, reporting likely results is sufficient for
encouraging discovery as results in question could be further explored
and checked using external approaches.

\subsection{Negative Relations and Definitions}

In addition to only providing probabilistic confidence for its
findings, there are some relations and definitions that are hard to
notice via a purely inductive, random-sampling based approach. For
instance, negative definitions such as learning that \texttt{scalene}
triangles are ones with \emph{no} equal sides would require the system
to handle more complicated logical combinations of relationships.

\subsection{Generality of Theorems}

Finally, the full space of theorems about geometry is quite broad.
Some of these statements require a richer set of tools than provided
in this system. For instance, noticing the fact that that ``the
shortest distance from a point to a line is along the perpendicular to
the line'' would require the current system to be testing and
searching for maxima and minima in its manipulations. The current
system is limited to discovering conjectures regarding simple
relationships among objects that are constructed from some initial
premises.

\section{System-level Extensions}

In addition to improvements to individual modules to reduce the
effects of randomness, filter out additional obvious properties, and
support more declarative constraints, there are several interesting
larger-scale extensions that could integrate with the system.

\subsection{Deductive Proof Systems}

One of the main extensions is to integrate the results from the system
with an automated, deductive geometry prover. Although such provers
often use less human-like approaches when verifying statements, having
access to such a system could increase this system's confidence in the
properties and conjectures it finds as it continues to explore new
concepts.

\subsection{Learning Constructions}

In addition to generating formal, deductive proofs about the
properties and theorems resulting from the system's explorations,
another interesting extension would be for the system to learn from
the \emph{process} it uses in generating its results. For example, the
sequence and dependencies for how values were determined in solving a
set of declarative constraints might be able to be abstracted into a
sequence of more typical construction procedures that produce the same
diagram.

\subsection{Self-directed Explorations}

A final exciting addition to the system is to build a self-directed
mode of operation in which the system proposes its own constructions
and diagrams to investigate rather than being prompted from an outside
user.  As the system expands its repository of knowledge about
constructions and conjectures, it could use these findings to direct
further explorations.  This would provide some full circle closure to
the discovery process and could even lead to the system creatively
devising interesting exercises or exam questions that test the
knowledge it has acquired.

\appendix
\chapter{Code Listings}
\label{chap:code}

This chapter contains code listings for the thesis presented in this
thesis. Code is implemented using MIT Scheme 9.2

\section{Repository Structure}

\section{External Dependencies}

GJS Propagator system, some scmutils, ghelper, eq-properties.


\lstlistoflistings

\lstdefinestyle{includes}{
  numbers=left,
  xleftmargin=0.3in,
  basicstyle=\fontsize{8pt}{9pt}\ttfamily,
}


\newcommand{\includecode}[2][c]{\lstinputlisting[caption=#2, style=includes]{code/#2}}
% ...
\linespread{1}

% Reset them at the end of this chapter!
\addtolength{\oddsidemargin}{-.4in}
\addtolength{\evensidemargin}{-.4in}
\addtolength{\textwidth}{0.8in}

\begin{landscape}
\begin{multicols}{2}
\includecode{load.scm}
\includecode{main.scm}

\includecode{figure/load.scm}
\includecode{figure/core.scm}
\includecode{figure/line.scm}
\includecode{figure/direction.scm}
\includecode{figure/vec.scm}
\includecode{figure/measurements.scm}
\includecode{figure/angle.scm}
\includecode{figure/bounds.scm}
\includecode{figure/circle.scm}
\includecode{figure/point.scm}
\includecode{figure/constructions.scm}
\includecode{figure/intersections.scm}
\includecode{figure/figure.scm}
\includecode{figure/math-utils.scm}
\includecode{figure/polygon.scm}
\includecode{figure/metadata.scm}
\includecode{figure/dependencies.scm}
\includecode{figure/randomness.scm}
\includecode{figure/transforms.scm}

\includecode{perception/load.scm}
\includecode{perception/observation.scm}
\includecode{perception/analyzer.scm}


\includecode{graphics/load.scm}
\includecode{graphics/appearance.scm}
\includecode{graphics/graphics.scm}

\includecode{manipulate/load.scm}
\includecode{manipulate/linkages.scm}
\includecode{manipulate/region.scm}
\includecode{manipulate/constraints.scm}
\includecode{manipulate/topology.scm}
\includecode{manipulate/mechanism.scm}
\includecode{manipulate/main.scm}

\includecode{learning/load.scm}
\includecode{learning/core-knowledge.scm}
\includecode{learning/lattice.scm}
\includecode{learning/definitions.scm}
\includecode{learning/conjecture.scm}
\includecode{learning/simplifier.scm}
\includecode{learning/student.scm}
\includecode{learning/walkthrough.scm}

\includecode{content/load.scm}
%\includecode{content/investigations.scm}
\includecode{content/thesis-demos.scm}

\includecode{core/load.scm}
\includecode{core/animation.scm}
\includecode{core/macros.scm}
\includecode{core/print.scm}
\includecode{core/utils.scm}

\includecode{lib/eq-properties.scm}
\includecode{lib/ghelper.scm}
% ...
\end{multicols}
\end{landscape}



% Reset Margins
\addtolength{\oddsidemargin}{.4in}
\addtolength{\evensidemargin}{.4in}
\addtolength{\textwidth}{-0.8in}

\clearpage
\newpage

%\include{appb}
\include{biblio}
\end{document}
