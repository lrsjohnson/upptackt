\chapter{Demonstration}
\label{chap:demo}

My system uses this idea of manipulating diagrams ``in the mind's
eye'' to explore and discover geometry theorems. Before discussing
some of the internal representations and modules, I will briefly
describe the goals of the system to provide direction and context to
understand the components.

\section{Imperative Figure Construction}

\begin{tcblisting}{colback=white,colframe=black!75!black,
listing only,
leftrule=3pt,
listing options={style=tcblatex,
basicstyle=\footnotesize\ttfamily}}
(define (triangle-with-pep-bisectors)
  (let-geo* ((a (make-point 0 0))
             (b (make-point 1.5 0))
             (c (make-point 1 1))
             (t (polygon-from-points a b c))
             (pb1 (perpendicular-bisector (make-segment a b)))
             (pb2 (perpendicular-bisector (make-segment b c)))
             (pb3 (perpendicular-bisector (make-segment c a))))
    (figure t pb1 pb2 pb3)))
\end{tcblisting}

\begin{tcblisting}{colback=white,colframe=black!75!black,
listing and comment, righthand width=5cm,
leftrule=3pt,
listing options={style=tcblatex,numbers=none,
basicstyle=\footnotesize\ttfamily},
tcbimage comment={images/demo-tri-pb-fig.png},
comment style={size=fbox,colframe=white,colback=white!50,height=8cm},
every listing line*={\textcolor{black}{\small\ttfamily\bfseries => }}}
(show-figure (triangle-with-perp-bisectors))
\end{tcblisting}



\begin{tcblisting}{colback=white,colframe=black!75!black,
listing only, righthand width=5cm,
leftrule=3pt,
fontlower=\ttfamily,
listing options={style=tcblatex,numbers=none,
basicstyle=\footnotesize\ttfamily},
comment style={size=fbox,colframe=white,colback=white!50,height=8cm}}
=> (show-figure (triangle-with-perp-bisectors))

((concurrent #[line 22] #[line 20] #[line 18])
 (perpendicular #[line 22] #[segment 21])
 (perpendicular #[line 20] #[segment 19])
 (perpendicular #[line 18] #[segment 17]))
\end{tcblisting}



\section{Declarative Constraint Solving}

\begin{lstlisting}[caption=Getting labels]
(define (arbitrary-triangle)
  (m:mechanism
   (m:establish-polygon-topology 'a 'b 'c)))
\end{lstlisting}

\begin{lstlisting}[caption=Constraint Solving for Isoceles Triangle]
(define (isoceles-triangle)
  (m:mechanism
   (m:establish-polygon-topology 'a 'b 'c)
   (m:c-length-equal (m:bar 'a 'b)
                     (m:bar 'b 'c))))
\end{lstlisting}


\begin{lstlisting}[caption=Constraint Solving for Isoceles Triangle]
(define (parallelogram-by-angles)
  (m:mechanism
   (m:establish-polygon-topology 'a 'b 'c 'd)
   (m:c-angle-equal (m:joint 'a)
                    (m:joint 'c))
   (m:c-angle-equal (m:joint 'b)
                    (m:joint 'd))))
\end{lstlisting}
