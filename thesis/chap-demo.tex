\chapter{Demonstration}
\label{chap:demo}

My system uses this idea of manipulating diagrams ``in the mind's
eye'' to explore and discover geometry theorems. Before discussing
some of the internal representations and modules, I will briefly
describe the goals of the system to provide direction and context to
understand the components.

\section{Imperative Figure Construction}

\begin{code-example}{Basic Example}
(define (triangle-with-pep-bisectors)
  (let-geo* ((a (make-point 0 0))
             (b (make-point 1.5 0))
             (c (make-point 1 1))
             (t (polygon-from-points a b c))
             (pb1 (perpendicular-bisector (make-segment a b)))
             (pb2 (perpendicular-bisector (make-segment b c)))
             (pb3 (perpendicular-bisector (make-segment c a))))
    (figure t pb1 pb2 pb3)))

(define (triangle-with-pep-bisectors)
  (let-geo* ((a (make-point 0 0))
             (b (make-point 1.5 0))
             (c (make-point 1 1))
             (t (polygon-from-points a b c))
             (pb1 (perpendicular-bisector (make-segment a b)))
             (pb2 (perpendicular-bisector (make-segment b c)))
             (pb3 (perpendicular-bisector (make-segment c a))))
    (figure t pb1 pb2 pb3)))
\end{code-example}

\begin{img-example}{Basic Example Image}{images/demo-tri-pb-fig.png}
=> (show-figure (triangle-with-perp-bisectors))
\end{img-example}

\begin{repl-example}{Simple Analysis}
=> (show-figure (triangle-with-perp-bisectors))

((concurrent #[line 22] #[line 20] #[line 18])
 (perpendicular #[line 22] #[segment 21])
 (perpendicular #[line 20] #[segment 19])
 (perpendicular #[line 18] #[segment 17]))
\end{repl-example}



\section{Declarative Constraint Solving}

\begin{img-example}{Arbitrary Triangle}{images/demo-arbitrary-tri.png}
Text Goes Here
\end{img-example}
\iffalse
\begin{code-example}{Demo}
(define (arbitrary-triangle)
  (m:mechanism
   (m:establish-polygon-topology 'a 'b 'c)))

=> (run-mechanism (arbitrary-triangle))

(specifying-joint m:joint:c:b:a .41203408293499)
(initializing-angle m:joint:c:b:a-dir-1 (direction 3.888926311421853))
(specifying-joint m:joint:a:c:b 1.8745808264593105)
(initializing-point m:bar:c:a-p1 (0 0))
(specifying-bar m:bar:c:a .4027149730292784)
\end{code-example}

\begin{repl-example}{Constraint Solving for Isoceles Triangle}
(define (isoceles-triangle)
  (m:mechanism
   (m:establish-polygon-topology 'a 'b 'c)
   (m:c-length-equal (m:bar 'a 'b)
                     (m:bar 'b 'c))))
\end{code-example}


\begin{code-example}{Constraint Solving for Isoceles Triangle}
(define (parallelogram-by-angles)
  (m:mechanism
   (m:establish-polygon-topology 'a 'b 'c 'd)
   (m:c-angle-equal (m:joint 'a)
                    (m:joint 'c))
   (m:c-angle-equal (m:joint 'b)
                    (m:joint 'd))))
\end{code-example}
\fi
