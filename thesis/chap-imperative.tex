\chapter{Imperative Construction System}
\label{chap:imperative}

\section{Overview}

The first module is an imperative system for performing geometry
constructions. This is the typical input method for generating
coordinate-backed instances of figures.

I first discuss the basic structures and constructions that comprise figures,
then describe the higher-order language used to specify construction
steps.

\section{Construction Interface}

\begin{code-example}{Sample Construction}
(define (angle-bisector-distance)
  (let-geo* (((a (r-1 v r-2)) (random-angle))
             (ab (angle-bisector a))
             (p (random-point-on-ray ab))
             ((s-1 (p b)) (perpendicular-to r-1 p))
             ((s-2 (p c)) (perpendicular-to r-2 p)))
     (figure a r-1 r-2 ab p s-1 s-2)))
\end{code-example}

\section{Basic Structures}

The basic structures

\begin{code-example}{Basic Structures}
(define-record-type <point>
  (make-point x y)
  point? "hi"
  (x point-x)
  (y point-y))

(define-record-type <segment>
  (%segment p1 p2)
  segment?
  (p1 segment-endpoint-1)
  (p2 segment-endpoint-2))

(define-record-type <line>
  (%make-line point dir)
  line?
  (point line-point) ;; Point on the line
  (dir line-direction))
\end{code-example}

\begin{code-example}{Angle and Circle Structures}
(define-record-type <angle>
  (make-angle dir1 vertex dir2)
  angle?
  (dir1 angle-arm-1)
  (vertex angle-vertex)
  (dir2 angle-arm-2))

(define-record-type <circle>
  (make-circle center radius)
  circle?
  (center circle-center)
  (radius circle-radius))
\end{code-example}

\subsection{Points}

Points form the basis of most elements. Throughout the system, points
are labeled and used to identify other elements.

\subsection{Linear Elements}

The linear elements of Segments, Lines, and Rays are built upon
points. Initially the internal representation of lines were that of
two points, but to simplify manipulations,

To better specify angles (see below), all linear elements, including
segments and lines are directioned. Thus, a line pointing. Predicates
exist that compare lines for equality ignoring

\subsection{Angles}

Initially angles were represented as three points, now vertex + two
directions. CCW orientation. Methods exist to determine them from
various pairs of linear elements, uses directionality of linear
elements to determine which ``quadrant'' of the angle is desired.

Given a figure, methods exist to extract angles from the diagrams in
analysis rather than specifying each angle of interest while creating
the diagram.

\subsection{Math Support}

Some ``core'' math structures to help these calculations: Direction
represents a direction in [0, 2pi], fixes principal value [0, 2pi],
and support various operations for direction intervals (basic
intersection, adding, shifting, etc). Currently all represented by
single theta value, could generalize via generics to dx, dy, or theta
depending on computation source.

\section{Higher-level structures}

In addition to the basic geometry structures, the system uses several
grouping structures to combine and abstract the basic figure elements
into higher-level figures elements.

For closure of combinators, all these higher level objects are also
``Diagram objects''.

\subsection{Polygons}

Polygons are represented as groups of points.

\subsection{Figures}

Figures are currently groups of elements. In the creation of figures
we extract additional information and build a graph out of adjacent
components for use in the analysis stages.

\section{Construction Operations}

\subsection{Traditional constructions}

Midpoint, perpendicular line, bisectors

\subsection{Intersections}

Generic intersections, mathematically based at line/line or
line/circle at the core. Other intersections also add the check that
the resulting point(s) are on the elements.

\subsection{Measurement-based operations}

A ``Ruler + Protractor'' is generally not permitted in traditional
construction problems. However, sometimes its nice to be able to use
measurements to more quickly compute a result (e.g.\ angle bisector by
halving angle) vs.\ going through the whole ray/circle based
construction process.

\subsection{Transformations}

Currently, rotate about a point or translate by a vector. Also
interfaces for by *random* point or vector.

\section{Randomness}

\subsection{Random Choices}

At the basis of all random

\subsection{Remembering choices}

\subsection{Backtracking}

Currently, the system does not backtrack based on random
choices. However, there are plans to perform checks on
randomly-generated elements that are too close to one another and to
retry the random choice to avoid degenerate choices.

\subsection{Avoiding almost-degenerate points}

As discussed above, randomly making choices in

\subsection{Animating choices}

I animate over a small range within the specified random
range. Top-level infrastructure determinies frames, sleeping, etc.
Constructions can request to animate functions of one arg [0, 1]. As
the figure and animation is run, each call to randomize gets a call to
random whenever their value is non-false.

\section{Dependencies}

\subsection{Implementation}

Eq-properties, etc.

\subsection{Naming}

Sometimes derived if unknown, figure out how name metadata relates to
the dependencies.

\subsection{Forcing higher-level random dependenceis}

"Inverts" the dependency tree that would otherwise usually go
down to points. set-dependency! as random-square. When given an
element by the teacher, generally we don't know how the construction
was performed.

\subsection{Dependency-less diagrams}

In some cases, the dependency structure of a figure can be wiped.

\section{Construction Language}

Constructions and instruction-based investigations are specified by
scheme procedures that return the desired figures.

\subsection{Macros}

I created a let-geo* special form that is similar to Scheme's (let
...) form, but sets the element names as specified so they can be more
easily referred to later.

\subsection{Multiple Assignment}

In let-geo*, I also permit some constructions to optionally map to
multiple assignments of names, such as the case in which you create a
triangle and simulatneously want to store and name the triangle's
vertex points.

\section{Graphics}

The system integrates with Scheme's graphics system for the X Window
System to display the figures for the users. The graphical viewer can
include labels and highlight specific elements, as well as display
animations representing the ``wiggling'' of the diagram.
