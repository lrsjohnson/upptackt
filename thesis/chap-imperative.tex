\chapter{Imperative Construction System}
\label{chap:imperative}

\section{Overview}

The first module is an imperative system for performing geometry
constructions. This is the typical input method for generating
coordinate-backed, analytic instances of figures.

The construction system is comprised of a large, versatile library of
useful utility and construction procedures for creating figures. To
appropriately focus the discussion of this module, I will concentrate
on the implementation of structures and procedures necessary for the
sample construction seen in Example \ref{sample-construction}.  Full
code and more usage examples are provided in Appendix \ref{chap:code}.

In doing so, I will first describe the basic structures and essential
utility procedures before presenting some higher-level construction
procedures, polygons, and figures. Then, I will Then, I will explore
the use of randomness in the system and examine how construction
language macros handle names, dependencies, and multiple assignment of
components.  Finally, I will briefly discuss the interface and
implementation for animating and displaying figures.

\begin{pdf-example}
[label=sample-construction]
{Sample Construction}
{images/angle-bisector-distance.pdf}
(define (angle-bisector-distance)
  (let-geo* (((a (r-1 v r-2)) (random-angle))
             (ab (angle-bisector a))
             (p (random-point-on-ray ab))
             ((s-1 (p b)) (perpendicular-to r-1 p))
             ((s-2 (p c)) (perpendicular-to r-2 p)))
     (figure a r-1 r-2 ab p s-1 s-2)))

=> (show-figure (angle-bisector-distance))
\end{pdf-example}

%% Outline
%% - Basic Structures and essential Utilities (code + line-from-points...)
%% - Randomness
%% - Higher-order Objects (Polygons, Figure)
%% - let-geo* Names and Dependencies
%% - Animation / Graphics.

\section{Basic Structures}

The basic structures in the imperative construction system are points,
segments, rays, lines, angles, and circles. These structures, as with
all structures in the system are implemented using Scheme record
structures as seen in Listings \ref{basic-structures} and
\ref{angle-circle-structs}. In the internal representations, lines and
segments are directioned. Predicates exist to allow other procedures
to work with or ignore these directions.

\begin{code-listing}
[label=basic-structures]
{Basic Structures}
(define-record-type <point>
  (make-point x y)
  point?
  (x point-x)
  (y point-y))

(define-record-type <segment>
  (%segment p1 p2)
  segment?
  (p1 segment-endpoint-1)
  (p2 segment-endpoint-2))

(define-record-type <line>
  (%make-line point dir)
  line?
  (point line-point) ;; Point on the line
  (dir line-direction))
\end{code-listing}

\begin{code-listing}
[label=angle-circle-structs]
{Angle and Circle Structures}
(define-record-type <angle>
  (make-angle dir1 vertex dir2)
  angle?
  (dir1 angle-arm-1)
  (vertex angle-vertex)
  (dir2 angle-arm-2))

(define-record-type <circle>
  (make-circle center radius)
  circle?
  (center circle-center)
  (radius circle-radius))
\end{code-listing}

\subsection{Creating Elements}

Elements can be created explicitly using the underlying
\texttt{make-*} constructors defined with the record types. However,
several higher-order constructors are provided to simplify
construction as shown in Listings~\ref{line-from-points}
and~\ref{angle-from}.  In \texttt{angle-from-lines}, we make use of
the fact that lines are directioned to uniquely specify an angle.  As
with the angle construction case, in several instances, we use generic
operations to handle mixed types of geometry elements.

\begin{code-listing}
[label=line-from-points]
{Higher-order Constructors}
(define (line-from-points p1 p2)
  (make-line p1 (direction-from-points p1 p2)))

\end{code-listing}

\begin{code-listing}
[label=angle-from]
{Generic Constructors for Creating Angles}
(define angle-from (make-generic-operation 2 'angle-from))

(define (angle-from-lines l1 l2)
  (let ((d1 (line->direction l1))
        (d2 (line->direction l2))
        (p (intersect-lines l1 l2)))
    (make-angle d1 p d2)))
(defhandler angle-from angle-from-lines line? line?)
\end{code-listing}

\subsection{Essential Math Utilities}

Several math utility structures support these constructors and other
geometry procedures. One particularly useful abstraction is a
\texttt{direction} that fixes a direction in the interval $[0, 2
  \pi]$. Listing~\ref{directions} provides a taste of some operations
using such abstractions.

\begin{code-listing}
[label=directions]
{Directions}
(define (subtract-directions d2 d1)
  (if (direction-equal? d1 d2)
      0
      (fix-angle-0-2pi (- (direction-theta d2)
                          (direction-theta d1)))))

(define (direction-perpendicular? d1 d2)
  (let ((difference (subtract-directions d1 d2)))
    (or (close-enuf? difference (/ pi 2))
        (close-enuf? difference (* 3 (/ pi 2))))))
\end{code-listing}

\section{Higher-order Procedures and Structures}

Higher-order construction procedures and structures are built upon
these basic elements and utilities. Listing~\ref{perpendiculars} shows
the implementation of the perpendicular constructions used in the
sample figure.

\begin{code-listing}
[label=perpendiculars]
{Perpendicular Constructions}
;; Constructs line through point perpendicular to linear-element
(define (perpendicular linear-element point)
  (let* ((direction (->direction linear-element))
         (rotated-direction (rotate-direction-90 direction)))
    (make-line point rotated-direction)))

;;; Constructs perpendicular segment from point to linear-element
(define (perpendicular-to linear-element point)
  (let ((pl (perpendicular linear-element point)))
    (let ((i (intersect-linear-elements pl (->line linear-element))))
      (make-segment point i))))
\end{code-listing}

Although traditional constructions generally avoid using rulers and
protractors, Listing~\ref{angle-bisector} shows the implementation of
the \texttt{angle-bisector} procedure from our sample figure that uses
measurements to simplify construction.

\begin{code-listing}
[label=angle-bisector]
{Angle Bisector Construction}
(define (angle-bisector a)
  (let* ((d2 (angle-arm-2 a))
         (vertex (angle-vertex a))
         (radians (angle-measure a))
         (half-angle (/ radians 2))
         (new-direction (add-to-direction d2 half-angle)))
    (make-ray vertex new-direction)))
\end{code-listing}

\subsection{Polygons and Figures}

Polygons record structures contain an ordered list of points in
counter-clockwise order, and provide procedures such as
\texttt{polygon-point-ref} or \texttt{polygon-segment} to obtain
particular points, segments, and angles specified by indices.

Figures are simple groupings of geometry elements and provide
procedures for extracting all points, segments, angles,
and lines contained in the figure, including ones extracted from within
polygons or subfigures.

\section{Random Choices}

To allow figures to represent general spaces of diagrams, random
choices are commonly used to instantiate diagrams. In our sample
figure, we use \texttt{random-angle} and \texttt{random-point-on-ray},
implementations of which are shown in
listing~\ref{random-angle-measure}. Underlying these procedures are
calls to Scheme's random function over a specified range ($[0, 2\pi]$
for \texttt{random-angle-measure}, for instance). Since infinite
ranges are not well supported and to ensure the figures stay
reasonable legible for a human viewer,
\texttt{extend-ray-to-max-segment} clips a ray at the current working
canvas so a point on the ray can be selected within the working
canvas.

\begin{code-listing}
[label=randomness]
{Random Constructors}
(define (random-angle)
  (let* ((v (random-point))
         (d1 (random-direction))
         (d2 (add-to-direction
              d1
              (rand-angle-measure))))
    (make-angle d1 v d2)))

(define (random-point-on-ray r)
  (random-point-on-segment
   (extend-ray-to-max-segment r)))

(define (random-point-on-segment seg)
  (let* ((p1 (segment-endpoint-1 seg))
         (p2 (segment-endpoint-2 seg))
         (t (safe-rand-range 0 1.0))
         (v (sub-points p2 p1)))
    (add-to-point p1 (scale-vec v t))))

(define (safe-rand-range min-v max-v)
  (let ((interval-size (max 0 (- max-v min-v))))
    (rand-range
     (+ min-v (* 0.1 interval-size))
     (+ min-v (* 0.9 interval-size)))))
\end{code-listing}

Other random elements are created by combining these random choices,
such as the random parallelogram in Listing
\ref{random-parallelogram}.

\begin{code-listing}
[label=random-parallelogram]
{Random Parallelogram}
(define (random-parallelogram)
  (let* ((r1 (random-ray))
         (p1 (ray-endpoint r1))
         (r2 (rotate-about (ray-endpoint r1)
                           (rand-angle-measure)
                           r1))
         (p2 (random-point-on-ray r1))
         (p4 (random-point-on-ray r2))
         (p3 (add-to-point
              p2
              (sub-points p4 p1))))
    (polygon-from-points p1 p2 p3 p4)))
\end{code-listing}

\subsection{Backtracking}

The module currently only provides limited support for avoiding
degenerate cases, or cases where randomly selected points happen to be
very nearly on top of existing points. Several random choices use
\texttt{safe-rand-range} (seen in Listing \ref{randomness}) to avoid
the edge cases of ranges, but further extensions could improve this
system to periodically check for unintended relationships and
backtrack to choose other values.

\section{Construction Language Support}

To simplify specification of figures, the module provides the
\texttt{leg-geo*} macro which allows for a multiple-assignment-like
extraction of components from elements and automatically tags
resulting elements with their variable names for future
reference. Listing~\ref{letgeo-expansion} shows the expansion of a
simple usage of \texttt{let-geo*} and
listing~\ref{component-assignment} shows some of the macros' implementation.

\begin{code-example}
[label=letgeo-expansion]
{Expansion of let-geo* macro}
(let-geo* (((a (r-1 v r-2)) (random-angle)))
  (figure a r-1 r-2 ...))

(let* ((a (random-angle))
       (r-1 (element-component a 0))
       (v   (element-component a 1))
       (r-2 (element-component a 2)))
  (set-element-name! a   'a)
  (set-element-name! r-1 'r-1)
  (set-element-name! v   'v)
  (set-element-name! r-2 'r-2)
  (figure a r-1 r-2 ...))
\end{code-example}

\begin{code-listing}
[label=component-assignment]
{Multiple and Component Assignment}
(define (expand-compound-assignment lhs rhs)
  (if (not (= 2 (length lhs)))
      (error "Malformed compound assignment LHS (needs 2 elements): " lhs))
  (let ((key-name (car lhs))
        (component-names (cadr lhs)))
    (if (not (list? component-names))
        (error "Component names must be a list:" component-names))
    (let ((main-assignment (list key-name rhs))
          (component-assignments (make-component-assignments
                                  key-name
                                  component-names)))
      (cons main-assignment
            component-assignments))))

(define (make-component-assignments key-name component-names)
  (map (lambda (name i)
         (list name `(element-component ,key-name ,i)))
       component-names
       (iota (length component-names))))
\end{code-listing}

Once expanded, a generic \texttt{element-component} operator shown in
Listing~\ref{element-component} defines what components are extracted
from what elements (endpoints for segments, vertices for polygons,
(ray, angle, ray) for angles).

\begin{code-listing}
[label=element-component]
{Generic Element Component Handlers}
(declare-element-component-handler polygon-point-ref polygon?)

(declare-element-component-handler
 (component-procedure-from-getters
  ray-from-arm-1
  angle-vertex
  ray-from-arm-2)
 angle?)
\end{code-listing}

\section{Graphics and Animation}

The system integrates with Scheme's graphics system for the X Window
System to display the figures for the users. The graphical viewer can
include labels and highlight specific elements, as well as display
animations representing the ``wiggling'' of the
diagram. Implementations of core procedures of these components are
shown in Listings~\ref{draw-figure} and~\ref{animation}.

\begin{code-listing}
[label=draw-figure]
{Drawing Figures}
(define (draw-figure figure canvas)
  (set-coordinates-for-figure figure canvas)
  (clear-canvas canvas)
  (for-each
   (lambda (element)
     (canvas-set-color canvas (element-color element))
     ((draw-element element) canvas))
   (all-figure-elements figure))
  (for-each
   (lambda (element)
     (canvas-set-color canvas (element-color element))
     ((draw-label element) canvas))
   (all-figure-elements figure))
  (graphics-flush (canvas-g canvas)))
\end{code-listing}

To animate a figure, constructions can call \texttt{animate} with a
procedure f that takes an argument in $[0, 1]$. When the animation is
run, the system will use fluid variables to iteratively wiggle each
successive random choice through its range of
$[0,1]$. \texttt{animate-range} provides an example where a user can
specify a range to wiggle over.

\begin{code-listing}
[label=animation]
{Animation}
(define (animate f)
  (let ((my-index *next-animation-index*))
    (set! *next-animation-index* (+ *next-animation-index* 1))
    (f (cond ((< *animating-index* my-index) 0)
             ((= *animating-index* my-index) *animation-value*)
             ((> *animating-index* my-index) 1)))))

(define (animate-range min max)
  (animate (lambda (v)
             (+ min
                (* v (- max min))))))
\end{code-listing}

%% Old Stuff:

\if false

\subsection{Points}

Points form the basis of most elements. Throughout the system, points
are labeled and used to identify other elements.

\subsection{Linear Elements}

The linear elements of Segments, Lines, and Rays are built upon
points. Initially the internal representation of lines were that of
two points, but to simplify manipulations,

To better specify angles (see below), all linear elements, including
segments and lines are directioned. Thus, a line pointing. Predicates
exist that compare lines for equality ignoring

\subsection{Angles}

Initially angles were represented as three points, now vertex + two
directions. CCW orientation. Methods exist to determine them from
various pairs of linear elements, uses directionality of linear
elements to determine which ``quadrant'' of the angle is desired.

Given a figure, methods exist to extract angles from the diagrams in
analysis rather than specifying each angle of interest while creating
the diagram.

\section{Higher-level structures}

In addition to the basic geometry structures, the system uses several
grouping structures to combine and abstract the basic figure elements
into higher-level figures elements.

For closure of combinators, all these higher level objects are also
``Diagram objects''.

\subsection{Polygons}

Polygons are represented as groups of points.

\subsection{Figures}

Figures are currently groups of elements. In the creation of figures
we extract additional information and build a graph out of adjacent
components for use in the analysis stages.

\section{Construction Operations}

\subsection{Traditional constructions}

\begin{code-listing}{Perpendiculars}
(define (perpendicular linear-element point)
  (let* ((direction (->direction linear-element))
         (rotated-direction (rotate-direction-90 direction)))
    (make-line point rotated-direction)))

;;; endpoint-1 is point, endpoint-2 is on linear-element
(define (perpendicular-to linear-element point)
  (let ((pl (perpendicular linear-element point)))
    (let ((i (intersect-linear-elements pl (->line linear-element))))
      (make-segment point i))))

(define (angle-bisector a)
  (let* ((d1 (angle-arm-1 a))
         (d2 (angle-arm-2 a))
         (vertex (angle-vertex a))
         (radians (angle-measure a))
         (half-angle (/ radians 2))
         (new-direction (add-to-direction d2 half-angle)))
    (make-ray vertex new-direction)))
\end{code-listing}

Midpoint, perpendicular line, bisectors

\subsection{Intersections}

Generic intersections, mathematically based at line/line or
line/circle at the core. Other intersections also add the check that
the resulting point(s) are on the elements.

\subsection{Measurement-based operations}

A ``Ruler + Protractor'' is generally not permitted in traditional
construction problems. However, sometimes its nice to be able to use
measurements to more quickly compute a result (e.g.\ angle bisector by
halving angle) vs.\ going through the whole ray/circle based
construction process.

\subsection{Transformations}

Currently, rotate about a point or translate by a vector. Also
interfaces for by *random* point or vector.

\section{Randomness}

\subsection{Random Choices}

At the basis of all random

\subsection{Remembering choices}

\subsection{Backtracking}

Currently, the system does not backtrack based on random
choices. However, there are plans to perform checks on
randomly-generated elements that are too close to one another and to
retry the random choice to avoid degenerate choices.

\subsection{Avoiding almost-degenerate points}

As discussed above, randomly making choices in

\subsection{Animating choices}

I animate over a small range within the specified random
range. Top-level infrastructure determinies frames, sleeping, etc.
Constructions can request to animate functions of one arg [0, 1]. As
the figure and animation is run, each call to randomize gets a call to
random whenever their value is non-false.

\section{Dependencies}

\subsection{Implementation}

Eq-properties, etc.

\subsection{Naming}

Sometimes derived if unknown, figure out how name metadata relates to
the dependencies.

\subsection{Forcing higher-level random dependenceis}

``Inverts'' the dependency tree that would otherwise usually go
down to points. Or set-dependency! as random-square. When given an
element by the teacher, generally we don't know how the construction
was performed.

\subsection{Dependency-less diagrams}

In some cases, the dependency structure of a figure can be wiped.

\section{Construction Language}

Constructions and instruction-based investigations are specified by
scheme procedures that return the desired figures.

\subsection{Macros}

I created a let-geo* special form that is similar to Scheme's (let
\ldots) form, but sets the element names as specified so they can be more
easily referred to later.

\subsection{Multiple Assignment}

In let-geo*, I also permit some constructions to optionally map to
multiple assignments of names, such as the case in which you create a
triangle and simulatneously want to store and name the triangle's
vertex points.

\section{Graphics}

The system integrates with Scheme's graphics system for the X Window
System to display the figures for the users. The graphical viewer can
include labels and highlight specific elements, as well as display
animations representing the ``wiggling'' of the diagram.

\begin{code-listing}{Drawing}
(define (draw-figure figure canvas)
  (set-coordinates-for-figure figure canvas)
  (clear-canvas canvas)
  (for-each
   (lambda (element)
     (canvas-set-color canvas (element-color element))
     ((draw-element element) canvas))
   (all-figure-elements figure))
  (for-each
   (lambda (element)
     (canvas-set-color canvas (element-color element))
     ((draw-label element) canvas))
   (all-figure-elements figure))
  (graphics-flush (canvas-g canvas)))
\end{code-listing}

\section{Discussion}

\fi
