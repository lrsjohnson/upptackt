\chapter{Perception Module}
\label{chap:observer}

\section{Overview}

Given a module that executes construction steps to build analytic
figures, we need a way of ``seeing'' these figures in our mind's eye.
Thus, the perception module is primarily concerned with the task of
examining the figure and observing interesting properties in figure.

\begin{code-listing}{Analyzer Routine}
(define (analyze figure)
  (let* ((points (figure-points figure))
         (angles (figure-angles figure))
         (linear-elements (figure-linear-elements figure))
         (segments (figure-segments figure)))
    (append
     (extract-relationships points
                            (list concurrent-points-relationship
                                  concentric-relationship
                                  concentric-with-center-relationship))
     (extract-relationships segments
                             (list equal-length-relationship))
     (extract-relationships angles
                             (list equal-angle-relationship
                                   supplementary-angles-relationship
                                   complementary-angles-relationship))
     (extract-relationships linear-elements
                             (list parallel-relationship
                                   concurrent-relationship
                                   perpendicular-relationship)))))
\end{code-listing}

\begin{code-listing}
(define (extract-relationship elements relationship)
  (map (lambda (tuple)
         (make-observation relationship tuple))
       (report-n-wise
        (relationship-arity relationship)
        (relationship-predicate relationship)
        elements)))

(define (report-n-wise n predicate elements)
  (let ((tuples (all-n-tuples n elements)))
    (filter (nary-predicate n predicate) tuples)))
\end{code-listing}

\subsection{Extracting segments and angles}

The observation module also builds and traverses a
graph-representation of the object of connectedness and adjacencies to
extract more segments and angles, or include intersections of elements
in its investigation.

\subsubsection{Auxillary Segments}

In some circumstances, the system can insert and consider segments
between all pairs of points. Although this can sometimes produce
interesting results, it can often lead to too many elements being
considered. This option is off by default but can be enabled in a
self-exploration mode.

\subsection{What is Interesting?}

Concurrent points, collinear points, equal angles,
supplementary/complementary angles, parallel, perpendicular elements,
concentric points, (future:) ratios between measurements, etc.

\begin{code-listing}{Relationships}
(define-record-type <relationship>
  (%make-relationship type arity predicate)
  relationship?
  (type relationship-type)
  (arity relationship-arity)
  (predicate relationship-predicate))

(define equal-length-relationship
  (%make-relationship 'equal-length 2 segment-equal-length?))

(define concurrent-relationship
  (%make-relationship 'concurrent 3 concurrent?))
\end{code-listing}

\subsection{Removing Obvious Properties}

This module makes use of available dependency information to eliminate
some obvious properties. At this phase, the eliminations arise only
from basic geometry knowledge ``hard-coded'' into the system, and not
upon any specific prior-learned formula.

\subsubsection{Trivial relations}

Points being on lines, segments, circles directly dependent on that point.

\subsubsection{Branch Relations}

Other examples include ``branch'' relations. [REF: Chen, Song,
  etc.]. ABCD on a line with AB = CD also means that AC = BD, for instance.

\section{Representations}

A ``Relationship'' object represents a type of relationship, a
``Observation'' object refers to a specific observation seen in a figure.
