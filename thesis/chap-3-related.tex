\chapter{Related Work}
\label{chap:related-work}

The topics of automating geometric proofs and working with diagrams
are areas of active research.  Several examples of related work can be
found in the proceedings of annual conferences such as \emph{Automated
  Deduction in Geometry} \cite{autoDeduction} and \emph {Diagrammatic
  Representation and Inference} \cite{diagramInference}.  In addition,
two papers from the past year combine these concepts with a layer of
computer vision interpretation of diagrams.  Chen, Song, and Wang
present a system that infers what theorems are being illustrated from
images of diagrams \cite{fromImages}, and a paper by Seo and
Hajishirzi describes using textual descriptions of problems to improve
recognition of their accompanying figures \cite{diagramUnderstanding}.

Further related work includes descriptions of the educational impacts
of dynamic geometry approaches and some software to explore geometric
diagrams and proofs.  However, such software typically uses alternate
approaches to automate such processes, and few focus on inductive
reasoning.

\section{Dynamic Geometry}
From an education perspective, there are several texts that emphasize
an investigative, conjecture-based approach to teaching such as
\emph{Discovering Geometry} by Michael Serra \cite{serraDiscovering},
the text I used to learn geometry.  Some researchers praise these
investigative methods \cite{geoTransformations} while others question
whether it appropriately encourages deductive reasoning skills
\cite{geoTeaching}.

\section{Software}
Some of these teaching methods include accompanying software such as
Cabri Geometry \cite{cabri} and the Geometer's Sketchpad
\cite{geoSketchpad} designed to enable students to explore
constructions interactively.  These programs occasionally provide
scripting features, but have no proof-related automation.

A few more academic analogs of these programs introduce some proof
features.  For instance, GeoProof \cite{geoProof} integrates diagram
construction with verified proofs using a number of symbolic methods
carried out by the Coq Proof Assistant, and Geometry Explorer
\cite{geoExplorer} uses a full-angle method of chasing angle relations
to check assertions requested by the user.  However, none of the
software described simulates the exploratory, inductive investigation
process used by students first discovering new conjectures.

\section{Automated Proof and Discovery}
Although there are several papers that describe automated discovery or
proof in geometry, most of these use alternate, more algebraic methods
to prove theorems.  These approaches include an area method
\cite{autoTools}, Wu's Method involving systems of polynomial
equations \cite{wuMethod}, and a system based on Gr\"obner Bases
\cite{grobner}.  Some papers discuss reasoning systems including the
construction and application of a deductive database of geometric
theorems \cite{deductiveDatabase}.  However, all of these methods
focused either on deductive reasoning or complex algebraic
reformulations.
