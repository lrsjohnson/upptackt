\chapter{Conclusion}
\label{chap:conclusion}

\section{Overview}

The system presented in this thesis provides a good foundation for
building, exploring, and analyzing geometry diagrams.

\section{Limitations}

Despite these successes, there are some limitations and

\subsection{Numerical Accuracy}

A big issue with any numerical analysis is dealing with numerical
accuracy. Use of \texttt{close-enuf?}

\subsection{Negative Relations and Definitions}

Negative properties are hard to determine: ``Scalene'' means not
isoceles or not equilateral, for instance.

\subsection{Generality of Theorems}

Similarly, some theorems involve more complicated statements ``the
shortest distance from a point to a line is along the perpendicular to
the line''. The current system is primarily focused around theorems
arising from simple premises.

\section{Extensions}

\subsection{Deductive Proof Systems}

Possible extensions include integrating with existing automated proof
systems (Coq, etc.)

\subsection{Learning Constructions}

Also: learning construction procedures from the declarative constraint
solver's solution.
