\chapter{Conclusion}
\label{chap:conclusion}

\section{Overview}

The system presented in this thesis provides a versatile framework for
building, exploring, and analyzing geometry diagrams. As shown in the
demonstrations, the modules can both be used independently to
construct and analyze interesting properties in geometric figures, and
combined with one another to discover new geometry concepts. By
constructing and examining figures, generalizing observations, solving
constraints, and aggregating results, the system has been able to
discover, learn, and simplify dozens of elementary geometry properties
and theorems.

In doing so, the process modeled and emulated the human-like process
of imagining and manipulating instance of problems ``in the mind's
eye'' to better understand new concepts. By focusing on noticing
interesting invariants in externally-specified investigations, it
simulates the effectiveness of an investigative-based approach to
learning and discovering geometry concepts.

Although the architecture of the four interrelated imperative
construction building, perceiving, declarative constraint solving, and
learning modules serves as a proof of concept of and foundation for
exploring such a learning approach, it has room for further
improvement and extension. Several chapters conclude with a discussion
section including ideas for future extensions and improvements.

In addition, while the techniques developed in this system generally
reflect my own approach to visualizing and thinking about geometry and
background in learning geometry via an investigative approach, there
is room to integrate the \emph{discovery} ideas in this system with
some of the techniques from the rich history of automated geometry
theorem \emph{proving}.

\section{Limitations}

Despite its successes, there are certainly limitations to the system's
current abilities. Reasoning about geometry concepts is a very broad
domain, and it becomes difficult to develop general techniques that
can apply in a wide variety of circumstances.
Chapters~\ref{chap:observer} and~\ref{chap:declarative} discuss how
this challenge arises when trying to filter more categories of obvious
observations and when deciding the ideal method for specifying values
in the constraint solver.  There are also some sizable limitations to
the system's purely-investigative approach that restrict what it is
able to discover:

\subsection{Probabilistic Approach}

One challenge is that its approach is inherently probabilistic. As
with any numerical-based system, an important issue with using a
coordinate-based, inductive technique for discovering concepts is
dealing with numerical inaccuracies. Although techniques were used to
lessen some of the effects of floating point errors, such techniques
also emphasize the probabilistic nature of the system.  Without using
deductive reasoning, the system cannot ever fully confirm its findings
are correct and may occasionally report false properties due to
uncertainty. However, reporting likely results is sufficient for
encouraging discovery as results in question could be further explored
and checked using external approaches.

\subsection{Negative Relations and Definitions}

In addition to only providing probabilistic confidence for its
findings, there are some relations and definitions that are hard to
notice via a purely inductive, random-sampling based approach. For
instance, negative definitions such as learning that \texttt{scalene}
triangles are ones with \emph{no} equal sides would require the system
to handle more complicated logical combinations of relationships.

\subsection{Generality of Theorems}

Finally, the full space of theorems about geometry is quite broad.
Some of these statements require a richer set of tools than provided
in this system. For instance, noticing the fact that that ``the
shortest distance from a point to a line is along the perpendicular to
the line'' would require the current system to be testing and
searching for maxima and minima in its manipulations. The current
system is limited to discovering conjectures regarding simple
relationships among objects that are constructed from some initial
premises.

\section{System-level Extensions}

In addition to improvements to individual modules to reduce the
effects of randomness, filter out additional obvious properties, and
support more declarative constraints, there are several interesting
larger-scale extensions that could integrate with the system.

\subsection{Deductive Proof Systems}

One of the main extensions is to integrate the results from the system
with an automated, deductive geometry prover. Although such provers
often use less human-like approaches when verifying statements, having
access to such a system could increase this system's confidence in the
properties and conjectures it finds as it continues to explore new
concepts.

\subsection{Learning Constructions}

In addition to generating formal, deductive proofs about the
properties and theorems resulting from the system's explorations,
another interesting extension would be for the system to learn from
the \emph{process} it uses in generating its results. For example, the
sequence and dependencies for how values were determined in solving a
set of declarative constraints might be able to be abstracted into a
sequence of more typical construction procedures that produce the same
diagram.

\subsection{Self-directed Explorations}

A final exciting addition to the system is to build a self-directed
mode of operation in which the system proposes its own constructions
and diagrams to investigate rather than being prompted from an outside
user.  As the system expands its repository of knowledge about
constructions and conjectures, it could use these findings to direct
further explorations.  This would provide some full circle closure to
the discovery process and could even lead to the system creatively
devising interesting exercises or exam questions that test the
knowledge it has acquired.
