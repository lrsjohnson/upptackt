\chapter{System Overview}
\label{chap:sys-overview}

My system uses this idea of manipulating diagrams ``in the mind's
eye'' to explore and discover geometry theorems. Before discussing
some of the internal representations and modules, I will briefly
describe the goals of the system to provide direction and context to
understand the components.

\section{Goals}

The end goal of the system is for it to be to notice and learn
interesting concepts in Geometry from inductive explorations.

Because these ideas are derived from inductive observation, we will
typically refer to them as conjectures. Once the conjectures are
reported, they can easily be integrated into existing automated proof
systems if a deductive proof is desired.

The conjectures explored in this system can be grouped into three
areas: definitions, properties, and theorems.

\begin{description}

\item[Properties] Properties include all the facts derived from a
  single premise. ``Opposite angles in a rhombus are equal'' or ``The
  midpoint of a segment divides it into two equal-length segments''.

\item[Definitions] Definitions classify and differentiate an object
  from other objects. For instance``What is a rhombus?'' yields the
  definition that it is a quadrilateral (classification) with four
  equal sides (differentiation). For definitions, the system will
  attempt to simplify definition properties to more minimal sets,
  provide alternative formations, and use pre-existing definitions
  when possible: ``A Square is a rhombus and a rectangle''

\item[Theorems] Theorems are very similar to properties but involve
  several premises. For instance, theorems about triangles may involve
  the construction of angle bisectors, incenters or circumcenters, or
  the interaction among several polygons in the same diagram.

\end{description}

Finally, given a repository of these conjectures about geometry, the
system will be able to apply its findings in future investigations by
examining elements to display its knowledge of definitions, and
focusing future investigations by omitting results implied by prior
theorems.

\section{Diagram Representations}

The system and modules are built around three core representations. As
discussed in the motivation section, we use the term ``diagram'' to
represent the abstract geometric object represented by these means:

\begin{description}

\item[Construction Steps] The main initial representation of most
  diagrams is a series of construction steps. These generally make up
  the input investigation from an external user trying to teach the
  system a concept. In some investigations, the actual construction
  steps are opaque to the system (as in a teacher that provides a
  process to ``magically'' produce rhombuses), but often, the
  construction steps use processes known by the system so that the
  resulting figures can include dependency information about how the
  figure was built.

\item[Analytic Figure] The second representation is an analytic figure
  for a particular instance of a diagram. This representation can be
  drawn and includes coordinates for all points in the diagram. This
  representation is used by the perception module to observe
  interesting relationships.

\item[Symbolic Relationships] Finally, the third representation is a
  collection of symbolic relationships or constraints on elements of
  the diagram. These are initially formed from the results of the
  perception module, but may also be introduced as known properties
  for certain premises and construction steps. These symbolic
  relationships can be further tested and simplified to discover which
  sets of constraints subsume one another.

\end{description}

While construction steps are primarily used as input and to generate
examples, as the system investigates a figure, the analytic figure and
symbolic relationship models get increasingly intertwined. The ``mind's
eye'' perception aspects of observing relationships in the analytic
figure lead to new symbolic relationships and a propagator-like
approach of wigging solutions to the symbolic constraints yields new
analytic figures.

As relationships are verified and simplified, results are output and
stored in the student's repository of geometry knowledge. This process
is depicted in the figure below and components are described in the
following chapters.


\newpage
\begin{center}
\includegraphics[width=0.9\textwidth]{diagrams/Representations.eps}
\end{center}

\noindent {\bf System Overview: Given construction steps for an
  investigation an external teacher wishes the student perform, the
  system first (1) uses its imperative construction module to
  execute these construction steps and build an analytic instance of
  the diagram. Then, (2) it will manipulate the diagram by
  ``wiggling'' random choices and use the perception module to observe
  interesting relationships. Given these relationships, it will (3)
  use the declarative propagator-based constraint solver to
  reconstruct a diagram satisfying a subset of the constraints to
  determine which are essential in the original diagram. Finally (4),
  a learning module will monitor the overall process, omit
  already-known results, and assemble a repository of known
  definitions, properties, and theorems.}

%% Other Stuff:

\subsection{Modules}

These four modules include an imperative geometry construction
interpreter used to build diagrams, a declarative geometry constraint
solver to solve and test specifications, an observation-based
perception module to notice interesting properties, and a learning
module to analyze information from the other modules and integrate it
into new definition and theorem discoveries.

\section{Sample Interaction}

This core system provides an interpreter to accept input of
construction instructions, an analytic geometry system that can create
instances of such constructions, a pattern-finding process to discover
``interesting properties'', and an interface for reporting findings.

\subsection{Interpreting Construction Steps}

The first step in such explorations is interpreting an input of the
diagram to be explored.  To avoid the problems involved with solving
constraint systems and the possibility of impossible diagrams, the
core system takes as input explicit construction steps that results in
an instance of the desired diagram.  These instructions can still
include arbitrary selections (let $P$ be some point on the line, or
let $A$ be some acute angle), but otherwise are restricted to basic
construction operations using a compass and straight edge.

To simplify the input of more complicated diagrams, some of these
steps can be abstracted into a library of known construction
procedures.  For example, although the underlying figures are be
limited to very simple objects of points, lines, and angles, the steps
of constructing a triangle (three points and three segments) or
bisecting a line or angle can be encapsulated into single steps.

\subsection{Creating Figures}

Given a language for expressing the constructions, the second phase of
the system is to perform such constructions to yield an instance of
the diagram.  This process mimics ``imagining'' manipulations and
results in an analytic representation of the figure with coordinates
for each point.  Arbitrary choices in the construction (``Let $Q$ be
some point not on the line.'') are chosen via an random process, but
with an attempt to keep the figures within a reasonable scale to ease
human inspection.

\subsection{Noticing Interesting Properties}
\label{sec:interest}

Having constructed a particular figure, the system examines it to find
interesting properties.  These properties involve facts that appear to
be ``beyond coincidence''.  This generally involves relationships
between measured values, but can also include ``unexpected''
configurations of points, lines, and circles.  As the system discovers
interesting properties, it will reconstruct the diagram using
different choices and observe if the observed properties hold true
across many instances of a diagram.

\subsection{Simplifying Definitions and Known Facts}

\subsection{Reporting Findings}

Finally, once the system has discovered some interesting properties
that appear repeatedly in instances of a given diagram, it reports its
results to the user via the learning module.  Although this includes a
simple list of all simple relationships, effort is taken to avoid
repeating observations that obvious in the construction.  For example,
if a perpendicular bisector of segment $AB$ is requested, the fact
that it bisects that segment in every instance is not informative.  To
do so, the construction process interacts with properties known in the
learning module to maintain a list of facts that can be reasoned from
construction assumptions so that these can be omitted in the final
reporting.

\section{Example Interaction}

[For now see walkthrough in the ``results'' chapter. Will add a good,
  simple example here]
