\chapter{Observation Module}
\label{chap:observer}

\section{Overview}

The observation module is primarily concerned with the task of
observing interesting properties in diagrams.

\subsection{Extracting more information}

The observation module also builds and traverses a
graph-representation of the object of connectedness and adjacencies to
extract more segments and angles, or include intersections of elements
in its investigation.

\subsubsection{Auxillary Segments}

In some circumstances, the system can insert and consider segments
between all pairs of points. Although this can sometimes produce
interesting results, it can often lead to too many elements being
considered. This option is off by default but can be enabled in a
self-exploration mode.

\subsection{What is Interesting?}

Concurrent points, collinear points, equal angles,
supplementary/complementary angles, parallel, perpendicular elements,
concentric points, (future:) ratios between measurements, etc.

\subsection{Removing Obvious Properties}

This module makes use of available dependency information to eliminate
some obvious properties. At this phase, the eliminations arise only
from basic geometry knowledge ``hard-coded'' into the system, and not
upon any specific prior-learned formula.

\subsubsection{Trivial relations}

Points being on lines, segments, circles directly dependent on that point.

\subsubsection{Branch Relations}

Other examples include ``branch'' relations. [TODO REF: Chen, Song,
  etc.]. ABCD on a line with AB = CD also means that AC = BD, for instance.
