\chapter{Declarative Geometry Constraint Solver}
\label{chap:declarative}

\section{Overview}

The third module is a declarative geometry constraint solver. Given a
user-specified topology of a diagram and various constraints on
segments and angles, this module attempts to solve the specification
by instantiating a figure that satisfies the constraints.

The solver is implemented using propagators, uses new types of partial
information about point regions and direction intervals, and focuses
on emulating the mental process of wiggling constrained figures in the
mind's eye. The physical nature of this process is captured by forming
analogies between geometry diagrams and mechanical linkages of bars
and joints.

After providing a brief overview of the mechanical analogies and quick
background on the propagator system, I examine an example of the
system solving a set of constraints for an under-constrained
rectangle. Then, I describe the module implementation, starting with
the new partial information representations and linkage constraints
before explaining how mechanisms are assembled and solved. Finally,
some limitations and extensions are discussed.

%% Outline
%% - Mechanical Analogies
%% - Propagator System
%% - Partial Information Structures
%% - Linkages
%% - Building a Mechanism
%% - Solving a Mechanism
%% - Discussion

\subsection{Mechanical Analogies}

Mechanical analogies are often applied to mathematical problems to
yield alternate, sometimes more-intuitive solutions. Several texts
such as [??], [??], and [??] study this and provide examples.

In this system, mechanical analogies are used to represent the physics
simulation going on as one mentally manipulates a diagram ``in the
mind's eye''. Often, given a diagram with constraints, one can imagine
assembling a physical example of the figure out of bars and joints in
one's head.  Some bars can be sliding to make their lengths adjustable
whereas others are constrained to be of equal length as one another.
As a person moves and wiggles these pieces to assemble satisfying
mechanisms, they can examine whether the resulting mechanisms retain
properties across instances and generalize such invariants into
theorems.

This module simulates this process by assembling mechanisms of bars
and joints, and using a propagator system to simulate incrementally
selecting where bars and joints are positioned while maintaining local
physical constraints.

\subsection{Propagator System}

The declarative geometry solver is built upon an existing propagator
system created by Alexey Radul under the advisement of Gerald Jay
Sussman \cite{gjs-propagator}. The propagator system allows a user to
create cells and connect them with propagator constraints. As content
is added to cells, their neighbors are notified and updated with
computations performed on the new information. Often, cells maintain a
representation of partial information about their content and merge
new information from several sources.

This module uses Radul's propagation system to handle the underlying
propagation of data, but implements constraints, partial information
types, specification protocols, and input formats particular to geometric
figures.

\section{Example of Solving Geometric Constraints}
\label{sec:example-solving}

I begin by fully explaining an example. The geometry problem of
inadequately constrained rectangles was introduced in the first
example of Chapter~\ref{chap:motivation} on page
\pageref{example-1}. The second proposed set of constraints in that
problem was expressed as a mechanism in Example~\ref{is-rect-2} in the
demonstration (page \pageref{is-rect-2}), and is repeated here in
Example~\ref{is-rect-specs}. Example~\ref{is-rect-solved} shows the
module's print messages as it solves the mechanism.

The illustrations in Explanation~\ref{illustration} and accompanying text
on the following pages explain how propagation is used to solve this
mechanism.

\begin{code-example}
[label=is-rect-specs]
{Rectangle Constraints Example}
(define (is-this-a-rectangle-2)
  (m:mechanism
   (m:establish-polygon-topology 'a 'b 'c 'd)
   (m:c-length-equal (m:bar 'a 'd) (m:bar 'b 'c))
   (m:c-right-angle (m:joint 'd))
   (m:c-angle-equal (m:joint 'a) (m:joint 'c))))
\end{code-example}

\begin{img-example}
[label=is-rect-solved]
{Solved Constraints}{images/rect-demo-2.png}
=> (m:run-mechanism (is-this-a-rectangle-2))

(specifying-bar m:bar:d:a .6742252545577186)
(initializing-direction m:bar:d:a-dir (direction 4.382829365403101))
(initializing-point m:bar:d:a-p1 (0 0))
(specifying-joint m:joint:c:b:a 2.65583669872538)
\end{img-example}

Solving a mechanism involves repeatedly selecting positions, lengths,
angles, and directions that are not fully specified and selecting
values within the domain of that element's current partial
information. As values are specified, the wiring of the propagator
model propagates further partial information to other values.

\refstepcounter{tcb@cnt@code-example}\label{illustration}
\begin{figure}[h]
\captionsetup{labelformat=empty}
\caption{{\bf Propagation Explanation
    \arabic{chapter}.\arabic{tcb@cnt@code-example}:} This series of
  illustrations depicts the propagation steps that occur to enable the
  system to solve the underconstrained rectangle from
  Example~\ref{is-rect-specs}.
}
\centering
\includegraphics[width=.95\textwidth]{diagrams/is-rect-explained-boards-1.eps}
\caption{{\bf Step 1:} The first value the module specifies is the
  length of bar \texttt{ad}. In doing so, it also initializes the
  bar's endpoint and direction to anchor it on the canvas. Because
  joint \texttt{d} is constrained to be a right angle, the system knows
  the direction but not length of bar \texttt{dc}. It propagates the
  partial information that point \texttt{c} is on the ray $r1$ extending
  out from \texttt{d} to the cell within point
  \texttt{c}. Furthermore, since bars \texttt{ad} and \texttt{bc} are
  constrained to have equal length, at this point, bar \texttt{bc} also knows
  its length but not direction. Next, the system specifies joint angle \texttt{b}:}
\includegraphics[width=.95\textwidth]{diagrams/is-rect-explained-boards-2.eps}
\caption{{\bf Step 2:} Once the angle measure of \texttt{b} is
  specified, constraints using the sum of angles in the specified
  polygon and a ``slice'' constraint on the pair of constrained angles
  will set the angle measures of joints \texttt{a} and \texttt{c} to
  be half of the remaining total:
  $\text{\texttt{a,c}}\protect\leftarrow\frac{2\protect\pi -
    \text{\texttt{b}} - \text{\texttt{d}}}{2}$. With these angles
  specified, point \texttt{b} is informed that it is on the ray $r2$
  and bar \texttt{bc} now knows both its length and direction.}
\vspace{-10em}
\end{figure}

\newpage
\begin{figure}[h]
\captionsetup{labelformat=empty}
\caption{{\bf Propagation Explanation
    \arabic{chapter}.\arabic{tcb@cnt@code-example} continued:} This series of
  illustrations depicts the propagation steps that occur to enable the
  system to solve the underconstrained rectangle solved in
  Example~\ref{is-rect-specs}.}
\centering
\includegraphics[width=.95\textwidth]{diagrams/is-rect-explained-boards-3.eps}
\caption{{\bf Step 3:} Since now both the length and direction of bar
  \texttt{bc} are known and point \texttt{c} is known to be on ray
  $r1$, the propagation constraints can translate this ray by the
  length and direction of \texttt{bc} and provide the information that
  point \texttt{b} must therefore also be on ray $r3$. This emulates
  the physical process of sliding bar \texttt{bc} along ray $r1$.}
\includegraphics[width=.95\textwidth]{diagrams/is-rect-explained-boards-4.eps}
\caption{{\bf Step 4:} The information about point \texttt{b} being on
  rays $r2$ and $r3$ is merged via ray intersection to fully determine
  the location of \texttt{b}. Then, once point \texttt{b} is
  specified, since the length and direction of bar \texttt{bc} is
  known, propagation sets the value and location of point \texttt{c},
  yielding a fully-specified solution.}
\vspace{-1.5em}
\end{figure}
Similar steps allow propagation to solve specifications for many
figures including isoceles triangles, parallelograms, and
quadrilaterals from their diagonals. Several of these are shown in
Section \ref{demo:sec:declarative}. In cases when bars have their
length and one endpoint specified first, the propagators specify that
the other endpoint is on an arc of a circle. The next sections
describe the implementation of these partial information structures
before explaining linkages and how mechanisms are built and solved.

\newpage
\section{Partial Information Structures}

Radul's propagation system typically used numeric intervals for
partial information. The declarative constraint solver uses standard
some numeric intervals, but also uses its own module-specific partial
information structures. These include \texttt{regions} and
\texttt{direction-intervals}, described below:

\subsection{Regions}

Regions are the partial information structure for point locations and
represent subsets of the plane where the points could be
located. These could be arbitrarily complex regions of the plane, but
the module currently implements point sets, rays, and arcs as shown in
Listing~\ref{regions}. As new information about locations are
provided, regions are merged by intersection. A contradiction region
represents an empty region.

\enlargethispage*{\baselineskip}
\begin{code-listing}
[label=regions]
{Region Structures}
(define-record-type <m:point-set>
  (%m:make-point-set points) ...)

(define-record-type <m:ray>
  (%m:make-ray endpoint direction) ...)

(define-record-type <m:arc>
  (m:make-arc center-point radius dir-interval) ...)

(define-record-type <m:region-contradiction>
  (m:make-region-contradiction error-regions) ...)

(defhandler merge m:intersect-regions m:region? m:region?)
\end{code-listing}

\subsection{Direction Intervals}

In addition, a module-specific direction interval structure is used
for the partial information about directions. Several additional
utilities were needed for working with and merging direction intervals
since directions form a periodic range $[0,2\pi)$. Currently, the
  subsystem treats an intersection of direction intervals that would
  yield multiple distinct direction intervals as providing no
  new information.

\section{Linkages and Basic Constraints}

The solver uses bar and joint linkages to represent segments and
angles. These structures are composed of propagator cells storing
information about locations, lengths, directions, and angles. To
assist with some of the propagation between these cells, the module
uses substructures for and vectors.

Point structures contain both numeric Cartesian coordinates and a cell
containing Region structures. The propagators \texttt{m:x-y->region}
and \texttt{m:region->x,y} transform location information between
these representations.

\begin{code-listing}
[label=point-region]
{Points and Regions}
(define (m:make-point)
  (let-cells (x y region)
    (p:m:x-y->region x y region)
    (p:m:region->x region x)
    (p:m:region->y region y)
    (%m:make-point x y region)))

(define (m:x-y->region x y)
  (m:make-singular-point-set (make-point x y)))
(propagatify m:x-y->region)

(define (m:region->x region)
  (if (m:singular-point-set? region)
      (point-x (m:singular-point-set-point region))
      nothing))
(propagatify m:region->x)
\end{code-listing}

Vectors represent the difference between two point and bidirectionally
propagate rectangular and polar information about the vector.

\begin{code-listing}
[label=vec-struct]
{Vectors}
(define (m:make-vec)
  (let-cells (dx dy length direction)
    (p:make-direction (e:atan2 dy dx) direction)
    (p:sqrt (e:+ (e:square dx)
                 (e:square dy))
            length)
    (p:* length (e:direction-cos direction) dx)
    (p:* length (e:direction-sin direction) dy)
    (%m:make-vec dx dy length direction)))
\end{code-listing}

\subsection{Bar Structure and Constraints}

As seen in Listing~\ref{bar-struct}, bar structure contains two
\texttt{m:point}s and a \texttt{m:vec} representing the distance and
direction between the points. The bar links these structures together
using simple bidirectional constraints on the coordinates. These
constrains will only propagate information when the bar's length and
direction are fully specified. \texttt{m:p1->p2-bar-propagator}
handles the other cases.

\begin{code-listing}
[label=bar-struct]
{Basic Bar Structure}
(define (m:make-bar bar-id)
  (let ((p1 (m:make-point))
        (p2 (m:make-point))
        (v (m:make-vec)))
    (c:+ (m:point-x p1) (m:vec-dx v)
         (m:point-x p2))
    (c:+ (m:point-y p1) (m:vec-dy v)
         (m:point-y p2))
    (let ((bar (%m:make-bar p1 p2 v)))
      (m:p1->p2-bar-propagator p1 p2 bar)
      (m:p2->p1-bar-propagator p2 p1 bar)
      bar)))
\end{code-listing}

The propagators specified by \texttt{m:p1->p2-bar-propagator} shown in
Listing~\ref{bar-propagator} propagate partial information about point
locations based on whether the bar's direction or length is
determined. \texttt{m:x-y-length-di->region} handles the case where
only the length of the bar is specifies and adds information to the
other endpoint's region cell that it is on the arc formed from the
bar's length and current direction interval.

\begin{code-listing}
[label=bar-propagator]
{Bar Region Propagator}
(define (m:p1->p2-bar-propagator p1 p2 bar)
  (let ((p1x (m:point-x p1))
        (p1y (m:point-y p1))
        (p1r (m:point-region p1))
        (p2r (m:point-region p2))
        (length (m:bar-length bar))
        (dir (m:bar-direction bar)))
    (p:m:x-y-direction->region p1x p1y dir p2r)
    (p:m:x-y-length-di->region p1x p1y length dir p2r)
    (p:m:region-length-direction->region p1r length dir p2r)))
(define (m:x-y-length-di->region px py length dir-interval)
  (if (direction-interval? dir-interval)
      (let ((vertex (make-point px py)))
        (m:make-arc vertex length dir-interval))
      nothing))
(propagatify m:x-y-length-di->region)
\end{code-listing}

\subsection{Joint Structure and Constraints}

Joints are represented by a vertex point, two directions, and an angle
representing the measure between the directions. Propagators express
this relationship between the direction. Special mechanism-specific
operators adding and subtracting directions were created since both
the direction and angle argument could be intervals. Creating
a joint also initializes its measure to the range $[0, \pi]$.

\begin{code-listing}
{Joint Constraints}
(define (m:make-joint)
  (let ((vertex (m:make-point)))
    (let-cells (dir-1 dir-2 theta)
      (p:m:add-to-direction dir-1 theta dir-2)
      (p:m:add-to-direction dir-2 (e:negate theta) dir-1)
      (p:m:subtract-directions dir-2 dir-1 theta)
      (m:instantiate theta (make-interval 0 *max-joint-swing*) 'theta)
      (%m:make-joint vertex dir-1 dir-2 theta))))
\end{code-listing}

\enlargethispage*{\baselineskip}
\section{User-specified Constraints}

In addition to constraints resulting from the bar and joint
connections, users can specify constraints on the
mechanism. Listing~\ref{constraint-struct} shows the structure for a
user constraint. These structures include a name, a list of bar or
joint identifiers the constraint applies to and a procedure used to
apply the constraint.

\begin{code-listing}
[label=constraint-struct]
{User Constraints}
(define-record-type <m:constraint>
  (m:make-constraint type args constraint-procedure) ...)
\end{code-listing}
\begin{code-listing}
[label=bar-eq]
{Bar Length Equality}
(define (m:c-length-equal bar-id-1 bar-id-2)
  (m:make-constraint
   'm:c-length-equal
   (list bar-id-1 bar-id-2)
   (lambda (m)
     (let ((bar-1 (m:lookup m bar-id-1))
           (bar-2 (m:lookup m bar-id-2)))
       (c:id (m:bar-length bar-1)
             (m:bar-length bar-2))))))
\end{code-listing}

This constraint procedure takes the assembled mechanism as its
argument. As shown in Listing~\ref{bar-eq}, such procedures
typically look up mechanism elements by bar or joint identifiers and
introduce additional constraints. In \texttt{m:c-length-equal}, the
lengths of the two bars are set to be identical to one another.

\subsection{Slice Constraints}

In addition to general user constraints, mechanisms are also support
slice constraints. These slices are structured in the manner as
constraints but are applied after all other user constraints, and thus
can use information about user constraints in adding their
propagators. In particular, the system uses slices to determine the
values of cells that are constrained as equal to one another within a
sum, once the total of the sum and all other cells in the sum have
been determined. This process is inspired by Gerald Jay Sussman's use
of slices to represent local patterns and help determine values in
propagation networks for circuit design \cite{gjs-slices}.

\section{Assembling Mechanisms}
\enlargethispage*{\baselineskip}

Mechanism structures in the declarative system are the analogs of
figures from the imperative system.  Here, instead of grouping
geometry elements, the mechanism group linkages and constraints. As
seen in Listing~\ref{mechanism-struct}, \texttt{m:mechanism} will
flatten and separate its arguments. Then, in addition to storing the
components in a record \texttt{m:make-mechanism} will also hash tables
for looking up bars and joints by their endpoint and vertex names.

\begin{code-listing}
[label=mechanism-struct]
{Mechanism Structure}
(define-record-type <m:mechanism>
    (%m:make-mechanism bars joints constraints slices
                        bar-table joint-table joint-by-vertex-table)...)

(define (m:mechanism . args)
  (let ((elements (flatten args)))
    (let ((bars (m:dedupe-bars (filter m:bar? elements)))
          (joints (filter m:joint? elements))
          (constraints (filter m:constraint? elements))
          (slices (filter m:slice? elements)))
      (m:make-mechanism bars joints constraints slices))))
\end{code-listing}

To assist with specifying the bars and joints for a closed polygon,
the utility \texttt{m:establish-polygon-topology} is often used. The
procedure takes $n$ vertex names as its arguments and returns $n$ bars
and $n$ joints. It uses the linkage constructors
\texttt{m:make-named-*} to attach names to the structures.  Such names
are later used to attach linkages to one another and to lookup
elements in constraint procedures.

\begin{code-listing}
[label=est-topo]
{Establishing Topology}
(define (m:establish-polygon-topology . point-names)
  (if (< (length point-names) 3)
      (error "Min polygon size: 3"))
  (let ((extended-point-names
         (append point-names (list (car point-names) (cadr point-names)))))
    (let ((bars (map (lambda (p1-name p2-name)
                       (m:make-named-bar p1-name p2-name))
                     point-names (cdr extended-point-names)))
          (joints (map (lambda (p1-name vertex-name p2-name)
                         (m:make-named-joint p1-name vertex-name p2-name))
                       (cddr extended-point-names)
                       (cdr extended-point-names)
                       point-names)))
      (append bars joints
              (list (m:polygon-sum-slice (map m:joint-name joints)))))))
\end{code-listing}

Once specified, mechanisms can be assembled using
\texttt{m:build-mechanism}.  That procedure first identifies all joint
vertices with the same names as being identical to one another to
handle topologies in which multiple joints share vertices. Then it
assembles bars and joints based on their names.

\begin{code-listing}
{Building Mechanisms}
(define (m:build-mechanism m)
  (m:identify-vertices m)
  (m:assemble-linkages (m:mechanism-bars m)
                       (m:mechanism-joints m))
  (m:apply-mechanism-constraints m)
  (m:apply-slices m))
\end{code-listing}

When assembling the mechanism, bars are identified into or out of the
arms of joints that share their names. Joints names refer to the three
vertices they connect and bar names refer to their two endpoint
vertices.  \texttt{m:identify-into-arm-1} demonstrates how bars and
joints get attached to one another. Corresponding point locations and
directions are constrained to be identical to one another via
\texttt{c:id}. Identifying two points involves identifying all of its
component properties.

\begin{code-listing}
[label=identifying-points]
{Identifying points}
(define (m:identify-into-arm-1 joint bar)
  (m:set-joint-arm-1 joint bar)
  (m:identify-points (m:joint-vertex joint) (m:bar-p2 bar))
  (c:id (ce:reverse-direction (m:joint-dir-1 joint))
        (m:bar-direction bar)))

(define (m:identify-points p1 p2)
  (for-each (lambda (getter)
              (c:id (getter p1) (getter p2)))
            (list m:point-x m:point-y m:point-region)))
\end{code-listing}

\section{Solving Mechanisms}

Once assembled, mechanisms can be solved via
\texttt{m:solve-mechanism}. Solving a mechanism involves repeatedly
selecting position, lengths, angles, and directions that are not fully
specified and selecting values within the domain of that element's
current partial information structure. As values are specified, the
constraint wiring of the propagator model propagates updated partial
information to other values.


\begin{code-listing}
[label=solve-mechanism]
{Solving Mechanisms}
(define (m:solve-mechanism m)
  (m:initialize-solve)
  (let lp ()
    (run)
    (cond ((m:mechanism-contradictory? m)
           (m:draw-mechanism m c)
           #f)
          ((not (m:mechanism-fully-specified? m))
           (if (m:specify-something m)
               (lp)
               (error "Couldn't find anything to specify.")))
          (else 'mechanism-built))))
\end{code-listing}

The ordering of what is specified is guided by a heuristic in
\texttt{m:specify-something}. This heuristic was determined
empirically and helps the majority the examples considered converge to
solutions. It generally prefers specifying the most constrained values
first. However, in some scenarios, specifying values in the wrong
order can yield premature contradictions. Additionally, sometimes
partial information about a value is incomplete and picking a value
arbitrarily may fail. A planned extension will attempt to recover from
such situations more gracefully by trying other values or orderings
for specifying components.

The system uses \texttt{m:instantiate} to add content to cells. As
seen in Listing~\ref{specify-something}, \texttt{m:instantiate} wraps
the value in a truth management system structure provided by Radul's
propagator system. These structures maintain dependencies for values, can report
which sets of premises are at odds with one another and allow
individual choices to be removed and replaced with new values.
\enlargethispage*{\baselineskip}
\begin{code-listing}
[label=specify-something]
{Specifying and Instantiating Values}
(define (m:specify-something m)
  (or
   (m:specify-bar-if m m:constrained?)
   (m:specify-joint-if m m:constrained?)
   (m:specify-joint-if m m:joint-anchored-and-arm-lengths-specified?)
   (m:initialize-bar-if m m:bar-length-specified?)
   ...)

(define (m:instantiate cell value premise)
  (add-content cell (make-tms (contingent value (list premise)))))
\end{code-listing}

\subsection{Interfacing with imperative diagrams}

Finally, as shown in Listing~\ref{to-figure},
\texttt{m:mechanism->figure} can convert fully specified mechanisms
into their corresponding figures so they can be observed and analyzed.

\begin{code-listing}
[label=to-figure]
{Converting to Figure}
(define (m:mechanism->figure m)
  (let ((points (map m:joint->figure-point (m:mechanism-joints m)))
        (segments (map m:bar->figure-segment (m:mechanism-bars m)))
        (angles (map m:joint->figure-angle (m:mechanism-joints m))))
    (apply figure (filter identity (append points segments angles)))))
\end{code-listing}

\section{Discussion and Extensions}

Future efforts involve an improved backtrack-search mechanism if
constraints fail, and a system of initializing the diagram with
content from an existing figure, kicking out and wiggling arbitrary
premises, and seeing how the resulting diagram properties respond.

\subsection{Backtracking}

If it can't build a figure with a given set of specifications, it will
first try some neighboring values, then backtrack and try a new value
for the previous element. After a number of failed attempts, it will
abort and claim that at this time, it is unable to build a diagram
satisfying the constraints.

%%% Extra:

\if false
\begin{code-listing}
[label=sum-slice]
{Sum Slice}
(define (m:equal-values-in-sum equal-cells all-cells total-sum)
  (let ((other-values (set-difference all-cells equal-cells eq?)))
    (c:id (car equal-cells)
          (ce:/ (ce:- total-sum (ce:multi+ other-values))
                (length equal-cells)))))

(define (m:sum-slice elements cell-transformer equality-predicate total-sum)
  (let* ((equivalence-classes
          (partition-into-equivalence-classes elements equality-predicate))
         (nonsingular-classes (filter nonsingular? equivalence-classes))
         (all-cells (map cell-transformer elements)))
    (cons (c:id total-sum (ce:multi+ all-cells))
          (map (lambda (equiv-class)
                 (m:equal-values-in-sum
                  (map cell-transformer equiv-class) all-cells total-sum))
               equivalence-classes))))
\end{code-listing}

\begin{code-listing}
[label=poly-sum-slice]
{Polygon Sum Slice}
(define (m:polygon-sum-slice all-joint-ids)
  (m:make-slice
   (m:make-constraint 'm:joint-sum all-joint-ids
    (lambda (m)
      (let ((all-joints (m:multi-lookup m all-joint-ids))
            (total-sum (n-gon-angle-sum (length all-joint-ids))))
        (m:joints-constrained-in-sum all-joints total-sum))))))

(define (m:joints-constrained-in-sum all-joints total-sum)
  (m:sum-slice all-joints m:joint-theta
   m:joints-constrained-equal-to-one-another? total-sum))
\end{code-listing}


\fi
