\chapter{Declarative Geometry Constraint Solver}
\label{chap:declarative}

\section{Overview}

The final module is a declarative geometric constraint solver. Given
a user-specified topology of the diagram and various constraints, this
system is able to solve those constraints and instantiate a diagram
that satisfies them if possible.

This system is implemented using propagators, involved the creation of
new partial information about point regions and direction intervals,
and focuses on a

Future efforts involve a backtrack-search mechanism if constraints
fail, and a system of initializing the diagram with content from an
existing figure, kicking out and wiggling arbitrary premises, and
seeing how the resulting diagram properties respond.

\section{Mechanical Analogies}

The geometry constraint solver - physical manipulation, simulation,
and ``wiggling''.

\section{Example}
\begin{code-example}{Rectangle Constraints Example}{images/rect-demo-2.png}
(define (is-this-a-rectangle-2)
  (m:mechanism
   (m:establish-polygon-topology 'a 'b 'c 'd)
   (m:c-length-equal (m:bar 'a 'd)
                     (m:bar 'b 'c))
   (m:c-right-angle (m:joint 'd))
   (m:c-angle-equal (m:joint 'a)
                    (m:joint 'c))))
\end{code-example}

\begin{img-example}{Solved Constraints}{images/rect-demo-2.png}
=> (m:run-mechanism (is-this-a-rectangle-2))

(m:run-mechanism rect-demo-2)
(specifying-bar m:bar:d:a .6742252545577186)
(initializing-direction m:bar:d:a-dir (direction 4.382829365403101))
(initializing-point m:bar:d:a-p1 (0 0))
(specifying-joint m:joint:c:b:a 2.65583669872538)
\end{img-example}


\subsection{Bar and Joint Linkages}

Bars have endpoints, directions and length. Joints have a vertex point
and two directions. Currently, most joints are directioned and have
max value of 180 degrees.

\subsection{Mechanism}

The Mechanism in our declarative system is analogous to Figure,
grouping elements. Also computes various caching and lookup tables to
more easily access elements.

\section{Partial Information}

\subsection{Regions}

Propagating partial information across bars and joints yields a new
region system: Regions include point sets of one or more possible
points, an entire ray, or an entire arc. These rays and arcs are
from an anchored bar with only one of direction or length specified,
for instance.

\subsection{Direction Intervals}

Ranges of intervals. Full circle + invalid intervals. Adding and
subtracting intervals of direction and thetas gets complicated at times.

Challenges with intersection, multiple segments. Eventually just
return nothing is okay.

\section{Propagator Constraints}

System uses propagators to solve these mechanism constraints.

\subsection{Basic Linkage Constraints}

Direction, dx, dy, length, thetas. ``Bars'' + ``Joints''

\subsection{Higher Order Constraints}

Angle sum of polygon, or scan through polygon and ensure that the
angles don't not match. Example is equilateral triangle, for
instance... Could also observe always ``60 degrees'' as an interesting
fact and put that in as a constraint. They're alebgraically quite
similar, but my propagators currently don't perform symbolic algebra.

\section{Solving: Specification Ordering}

Given a wired diagram, process is repeatedly specifying values for elements

\subsection{Anchored vs. Specified vs. Initialized}

\section{Backtracking}

If it can't build a figure with a given set of specifications, it will
first try some neighboring values, then backtrack and try a new value
for the previous element. After a number of failed attempts, it will
abort and claim that at this time, it is unable to build a diagram
satisfying the constraints.

(This doesn't mean that it is impossible: Add analysis/info about what
it can/can't solve)

\section{Interfacing with existing diagrams}

Converts between figures and symbolic relationships.

\section{Specification Interface}

\subsubsection{Establish Polygon Topology}

Nice techniques for establishing polygon topology.
