\chapter{Related Work}
\label{chap:related-work}

The topics of working with geometry theorems and diagrams have rich
histories yet are still areas of active research.

As a seminal paper in the field, in the early 1960s, Herbert Gelernter
created a ``Geometry Theorem Proving Machine'' \cite{gelernter}. His
machine focused on a deductive process to search for proofs and used a
formal system based on strings of characters. In addition to purely
logic-based inference rules, the system also asks the user requesting
a proof to provide a coordinate-backed diagram against which the
system checks various subgoals it is considering in a proof.

Despite this long history, several examples of related work are still
found in the proceedings of annual conferences such as \emph{Automated
  Deduction in Geometry} \cite{autoDeduction} and \emph {Diagrammatic
  Representation and Inference} \cite{diagramInference}.  In addition,
two papers from the past year combine these concepts with a layer of
computer vision interpretation of diagrams.  Chen, Song, and Wang
present a system that infers what theorems are being illustrated from
images of diagrams \cite{fromImages}, and a paper by Seo and
Hajishirzi describes using textual descriptions of problems to improve
recognition of their accompanying figures \cite{diagramUnderstanding}.

The main areas of work related to my thesis are automated geometry
theorem proof, automated geometry theorem discovery, and mechanical
analogs of geometry concepts. After explaining some systems in these
areas, I will discuss further related work including descriptions of
the educational impacts of dynamic geometry approaches and some
software to explore geometric diagrams and proofs.

Some systems use techniques similar to those in this system's modules,
but most approaches focus on deductive proof or complicated algebraic
reformulations rather than inductive reasoning and exploration.

\section{Automated Geometry Proof}

As opposed to my system which focuses on modeling a student's
investigations and \emph{discoveries} about geometry, the main focus
of historic Artificial Intelligence efforts related to geometry was
obtaining \emph{proofs} for theorems given by a user. Projects
explored both algebraic and synthetic approaches, some of which
involved using diagrams in addition to purely symbolic manipulations
\cite{chou1988mechanical}, \cite{goldstein1973elementary},
\cite{nevins1975plane}. Texts such as \cite{jamnik2001mathematical}
include a more detailed history and description of such systems. These
systems are reasonably powerful but generally produce long proofs.

\section{Automated Geometry Discovery}
Several papers also describe automated \emph{discovery} in
geometry. However, most of these use alternate, more algebraic methods
to find and later prove theorems.  These approaches include an area
method \cite{autoTools}, Wu's Method involving systems of polynomial
equations \cite{wuMethod}, and a system based on Gr\"obner Bases
\cite{grobner}.  Some papers discuss reasoning systems including the
construction and application of a deductive database of geometric
theorems \cite{deductiveDatabase}.  However, all of these methods
focused on either deductive reasoning or complex algebraic
reformulations.

The effort closest to my system's approach is Chen, Song and Wang's
``Automated Generation of Geometric Theorems from Images of Diagrams''
\cite{fromImages}. This paper includes an initial section with several
image processing algorithms for detecting points and segments from
images. It then applies a series of heuristic strategies to determine
which elements are particularly relevant and propose candidate
theorems. These strategies generally involved assigning weights to
points to determine which are ``characteristic points'' or ``points of
attraction.'' By doing so, their system successfully proposed several
nontrivial theorems that the original image could have been
illustrating.  Integrating some of these strategies into my system
would be an interesting extension.

\section{Geometry Constraint Solving and Mechanics}

Ideas about solving geometry diagram constraints are related to the
fields of kinematic mechanisms and computer-aided design. Glenn
Kramar provides a system for solving geometry constraints in
mechanisms \cite{kramer1992solving}, but focuses on several practical
three-dimensional case studies with complicated joints. Summaries such
as \cite{joan2009basics} provide more information about other
graph-based, logic-based, and algebraic methods for solving 2D
geometry constraints. My system builds on a propagator system by
Alexey Radul and Gerald Jay Sussman \cite{radul-propagator} and
applies it to simple geometry constraints.

\section{Dynamic Geometry}
From an education perspective, there are several texts that emphasize
an investigative, conjecture-based approach to teaching. These include
\emph{Discovering Geometry} by Michael Serra \cite{serraDiscovering},
the text I used to learn geometry and that served as an inspiration to
this thesis project.  Some researchers praise these investigative
methods \cite{geoTransformations} while others question whether they
appropriately encourage deductive reasoning skills \cite{geoTeaching}.

\section{Software}
Some of these teaching methods include accompanying software such as
Cabri Geometry \cite{cabri} and the Geometer's Sketchpad
\cite{geoSketchpad} designed to enable students to explore
constructions interactively.  These programs occasionally provide
scripting tools, but have no theorem or proof-related automation.

A few more academic analogs of these programs introduce some proof
features.  For instance, GeoProof \cite{geoProof} integrates diagram
construction with verified proofs using a number of symbolic methods
carried out by the Coq Proof Assistant, and Geometry Explorer
\cite{geoExplorer} uses a full-angle method of chasing angle relations
to check assertions requested by the user.  However, none of the
software described simulates or automates the exploratory, inductive
investigation process used by students first discovering new
conjectures.

One interesting piece of software is Geometer \cite{geometer} created
by Tom Davis. Like the other programs, Geometer is primarily a user
interface for accurately constructing diagrams. It does not attempt to
produce or prove theorems, but does have a ``Test Diagram'' mode. When
this mode is activated, the user can wiggle elements in the diagram as
they please. When ``End Test'' is selected, the program lists all
features that were maintained during the users' manipulations. The
creator claims that these observations can be useful pieces for a user
attempting to deductively prove a theorem about the figure they are
drawing. This is a similar to the observations and manipulations in my
system but requires the user to manually manipulate elements in the
figure rather than automatically arbitrary choices in a specified
construction.
