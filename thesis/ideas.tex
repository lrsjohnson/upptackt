% Created 2015-04-30 Thu 14:13
\documentclass[11pt]{article}
\usepackage[utf8]{inputenc}
\usepackage[T1]{fontenc}
\usepackage{fixltx2e}
\usepackage{graphicx}
\usepackage{longtable}
\usepackage{float}
\usepackage{wrapfig}
\usepackage{soul}
\usepackage{textcomp}
\usepackage{marvosym}
\usepackage{wasysym}
\usepackage{latexsym}
\usepackage{amssymb}
\usepackage{hyperref}
\tolerance=1000
\providecommand{\alert}[1]{\textbf{#1}}

\title{Ideas}
\author{Lars Johnson}
\date{\today}
\hypersetup{
  pdfkeywords={},
  pdfsubject={},
  pdfcreator={Emacs Org-mode version 7.9.3f}}

\begin{document}

\maketitle

\section{Overview}
\label{sec-1}
\subsection{Visualizing ``In the Mind's Eye''}
\label{sec-1-1}
\section{Prior Work}
\label{sec-2}
\section{Imperative Geometry Construction System}
\label{sec-3}
\subsection{Basic Structures}
\label{sec-3-1}
\subsubsection{Points}
\label{sec-3-1-1}
\subsubsection{Segments, Lines, Rays}
\label{sec-3-1-2}
\subsubsection{Angles}
\label{sec-3-1-3}

\begin{itemize}
\item Initially three points, now vertext + two direction
\item CCW orientation
\item Methods to determine from elements
\item ``Arms'' of an angle: ``Directions''
\item[Future:] Possibly automatically extract angles from diagrams
      (e.g. from a pile of points and segments)
\end{itemize}
\subsubsection{Circles, Arcs}
\label{sec-3-1-4}
\subsubsection{Math Support (Directions, Vectors)}
\label{sec-3-1-5}
\subsection{Construction Operations}
\label{sec-3-2}
\subsubsection{Traditional}
\label{sec-3-2-1}
\subsubsection{Measurement-based}
\label{sec-3-2-2}
\subsubsection{Transformations}
\label{sec-3-2-3}
\subsection{Randomness}
\label{sec-3-3}
\subsubsection{Remembering choices}
\label{sec-3-3-1}
\subsubsection{Avoiding Almost-degenerate points}
\label{sec-3-3-2}
\subsubsection{Animating}
\label{sec-3-3-3}
\subsection{Dependencies}
\label{sec-3-4}
\subsubsection{Naming}
\label{sec-3-4-1}
\subsubsection{Dependencies}
\label{sec-3-4-2}
\subsubsection{Forcing higher-level random dependencies}
\label{sec-3-4-3}
\subsection{Construction Language}
\label{sec-3-5}
\subsubsection{Macros}
\label{sec-3-5-1}
\subsubsection{Multiple Assignment}
\label{sec-3-5-2}
\subsection{Analysis}
\label{sec-3-6}
\subsubsection{What is Interesting}
\label{sec-3-6-1}
\subsubsection{Removing obvious properties}
\label{sec-3-6-2}
\section{Declarative Geometry Constraint Solver}
\label{sec-4}
\subsection{Mechanical Analogies}
\label{sec-4-1}
\subsubsection{Bars, Joint Linkages}
\label{sec-4-1-1}
\subsubsection{Mechanism}
\label{sec-4-1-2}
\subsubsection{Propagators}
\label{sec-4-1-3}
\subsection{Partial Information}
\label{sec-4-2}
\subsubsection{Regions}
\label{sec-4-2-1}
\subsubsection{Direction Intervals}
\label{sec-4-2-2}
\subsection{Propagator Constraints}
\label{sec-4-3}
\subsubsection{Basic Constraints for Bar, Joint}
\label{sec-4-3-1}
\subsubsection{Higher-level constraints}
\label{sec-4-3-2}
\subsection{Specification Ordering}
\label{sec-4-4}
\subsubsection{Constraints First}
\label{sec-4-4-1}
\section{Learning}
\label{sec-5}

\end{document}
