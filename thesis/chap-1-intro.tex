%% This is an example first chapter.  You should put chapter/appendix that you
%% write into a separate file, and add a line \include{yourfilename} to
%% main.tex, where `yourfilename.tex' is the name of the chapter/appendix file.
%% You can process specific files by typing their names in at the
%% \files=
%% prompt when you run the file main.tex through LaTeX.
\chapter{Introduction}
\label{chap:intro}

In this thesis, I develop and analyze an interactive computer system
that emulates a student learning geometry concepts through inductive
investigation. Although geometry knowledge can be conveyed via a
series of factual definitions, theorems, and proofs, my system focuses
on a more investigative approach in which an external teacher guides
the student to ``discover'' new definitions and theorems via
explorations and self-directed inquiry.

My system emulates such a student by beginning with a fairly limited
knowledge set regarding basic definitions in geometry and providing a
means by which a user interacting with the system can ``teach''
additional geometric concepts and theorems by suggesting
investigations the system should explore to see if it ``notices
anything interesting.''

To enable such learning, my project includes the combination of four
intertwined modules: an imperative geometry construction interpreter
to build constructions, a declarative geometry constraint solver to
solve and test specifications, an observation-based perception module
to notice interesting properties, and a learning module to analyze
information from the other modules and integrate it into new
definition and theorem discoveries.

To evaluate its recognition of such concepts, my system provides means
for a user to extract the observations and apply its findings to new
scenarios.  Through a series of simple investigations similar to an
introductory course in geometry, the system [\emph{not quite yet, but
    hopefully}] has been able to propose and learn a few dozen
standard geometry theorems, and through more self-directed
explorations, it has discovered several interesting properties and
theorems not typically covered in standard mathematics courses.

\section{Document Structure}

\begin{description}

\item [Chapter~\ref{chap:motivation}] further discusses motivation of
  the system and presents some examples of diagram manipulation,
  emphasizing the technique of visualizing diagrams ``in the mind's
  eye.''

\item[Chapter~\ref{chap:related-work}] discusses some related work to
  automated geometry theorem discovery and proof, as well as a
  comparison with existing dynamic geometry systems.

\item[Chapter~\ref{chap:sys-overview}] further introduces the system
  modules and discusses how they work together in the discovery of new
  definitions and theorems.

\item[Chapter~\ref{chap:imperative}] describes the implementation and
  function of the imperative construction module that enables the
  system to carry out constructions.

\item[Chapters~\ref{chap:declarative}] describes the implementation
  and function of the propagator-based declarative geometry constraint
  solver that builds instances of diagrams satisfying declarative
  constraints.

\item[Chapter~\ref{chap:observer}] describes the implementation and
  function of the perception module focused on observing interesting
  properties in diagrams. A key question involves determining ``what
  is interesting''.

\item[Chapter~\ref{chap:analyzer}] describes the analyzer module which
  integrates results from the other systems to create new
  discoveries. Main features include filtering out obvious or known
  results to focus on the most interesting discoveries, the
  persistence and storage of definitions and theorems, and an
  interface to apply these findings to new situations.

\item[Chapter~\ref{chap:results}] discusses several of the definitions
  and theorems results the overall system has been able to discover
  and learn.

\item[Chapter~\ref{chap:conclusion}] discusses the strengths and
  weaknesses of the system. Future work and possible extensions are
  discussed.

\end{description}
